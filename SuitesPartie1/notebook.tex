
% Default to the notebook output style

    


% Inherit from the specified cell style.




    
\documentclass[11pt]{article}

    
    
    \usepackage[T1]{fontenc}
    % Nicer default font (+ math font) than Computer Modern for most use cases
    \usepackage{mathpazo}

    % Basic figure setup, for now with no caption control since it's done
    % automatically by Pandoc (which extracts ![](path) syntax from Markdown).
    \usepackage{graphicx}
    % We will generate all images so they have a width \maxwidth. This means
    % that they will get their normal width if they fit onto the page, but
    % are scaled down if they would overflow the margins.
    \makeatletter
    \def\maxwidth{\ifdim\Gin@nat@width>\linewidth\linewidth
    \else\Gin@nat@width\fi}
    \makeatother
    \let\Oldincludegraphics\includegraphics
    % Set max figure width to be 80% of text width, for now hardcoded.
    \renewcommand{\includegraphics}[1]{\Oldincludegraphics[width=.8\maxwidth]{#1}}
    % Ensure that by default, figures have no caption (until we provide a
    % proper Figure object with a Caption API and a way to capture that
    % in the conversion process - todo).
    \usepackage{caption}
    \DeclareCaptionLabelFormat{nolabel}{}
    \captionsetup{labelformat=nolabel}

    \usepackage{adjustbox} % Used to constrain images to a maximum size 
    \usepackage{xcolor} % Allow colors to be defined
    \usepackage{enumerate} % Needed for markdown enumerations to work
    \usepackage{geometry} % Used to adjust the document margins
    \usepackage{amsmath} % Equations
    \usepackage{amssymb} % Equations
    \usepackage{textcomp} % defines textquotesingle
    % Hack from http://tex.stackexchange.com/a/47451/13684:
    \AtBeginDocument{%
        \def\PYZsq{\textquotesingle}% Upright quotes in Pygmentized code
    }
    \usepackage{upquote} % Upright quotes for verbatim code
    \usepackage{eurosym} % defines \euro
    \usepackage[mathletters]{ucs} % Extended unicode (utf-8) support
    \usepackage[utf8x]{inputenc} % Allow utf-8 characters in the tex document
    \usepackage{fancyvrb} % verbatim replacement that allows latex
    \usepackage{grffile} % extends the file name processing of package graphics 
                         % to support a larger range 
    % The hyperref package gives us a pdf with properly built
    % internal navigation ('pdf bookmarks' for the table of contents,
    % internal cross-reference links, web links for URLs, etc.)
    \usepackage{hyperref}
    \usepackage{longtable} % longtable support required by pandoc >1.10
    \usepackage{booktabs}  % table support for pandoc > 1.12.2
    \usepackage[inline]{enumitem} % IRkernel/repr support (it uses the enumerate* environment)
    \usepackage[normalem]{ulem} % ulem is needed to support strikethroughs (\sout)
                                % normalem makes italics be italics, not underlines
    

    
    
    % Colors for the hyperref package
    \definecolor{urlcolor}{rgb}{0,.145,.698}
    \definecolor{linkcolor}{rgb}{.71,0.21,0.01}
    \definecolor{citecolor}{rgb}{.12,.54,.11}

    % ANSI colors
    \definecolor{ansi-black}{HTML}{3E424D}
    \definecolor{ansi-black-intense}{HTML}{282C36}
    \definecolor{ansi-red}{HTML}{E75C58}
    \definecolor{ansi-red-intense}{HTML}{B22B31}
    \definecolor{ansi-green}{HTML}{00A250}
    \definecolor{ansi-green-intense}{HTML}{007427}
    \definecolor{ansi-yellow}{HTML}{DDB62B}
    \definecolor{ansi-yellow-intense}{HTML}{B27D12}
    \definecolor{ansi-blue}{HTML}{208FFB}
    \definecolor{ansi-blue-intense}{HTML}{0065CA}
    \definecolor{ansi-magenta}{HTML}{D160C4}
    \definecolor{ansi-magenta-intense}{HTML}{A03196}
    \definecolor{ansi-cyan}{HTML}{60C6C8}
    \definecolor{ansi-cyan-intense}{HTML}{258F8F}
    \definecolor{ansi-white}{HTML}{C5C1B4}
    \definecolor{ansi-white-intense}{HTML}{A1A6B2}

    % commands and environments needed by pandoc snippets
    % extracted from the output of `pandoc -s`
    \providecommand{\tightlist}{%
      \setlength{\itemsep}{0pt}\setlength{\parskip}{0pt}}
    \DefineVerbatimEnvironment{Highlighting}{Verbatim}{commandchars=\\\{\}}
    % Add ',fontsize=\small' for more characters per line
    \newenvironment{Shaded}{}{}
    \newcommand{\KeywordTok}[1]{\textcolor[rgb]{0.00,0.44,0.13}{\textbf{{#1}}}}
    \newcommand{\DataTypeTok}[1]{\textcolor[rgb]{0.56,0.13,0.00}{{#1}}}
    \newcommand{\DecValTok}[1]{\textcolor[rgb]{0.25,0.63,0.44}{{#1}}}
    \newcommand{\BaseNTok}[1]{\textcolor[rgb]{0.25,0.63,0.44}{{#1}}}
    \newcommand{\FloatTok}[1]{\textcolor[rgb]{0.25,0.63,0.44}{{#1}}}
    \newcommand{\CharTok}[1]{\textcolor[rgb]{0.25,0.44,0.63}{{#1}}}
    \newcommand{\StringTok}[1]{\textcolor[rgb]{0.25,0.44,0.63}{{#1}}}
    \newcommand{\CommentTok}[1]{\textcolor[rgb]{0.38,0.63,0.69}{\textit{{#1}}}}
    \newcommand{\OtherTok}[1]{\textcolor[rgb]{0.00,0.44,0.13}{{#1}}}
    \newcommand{\AlertTok}[1]{\textcolor[rgb]{1.00,0.00,0.00}{\textbf{{#1}}}}
    \newcommand{\FunctionTok}[1]{\textcolor[rgb]{0.02,0.16,0.49}{{#1}}}
    \newcommand{\RegionMarkerTok}[1]{{#1}}
    \newcommand{\ErrorTok}[1]{\textcolor[rgb]{1.00,0.00,0.00}{\textbf{{#1}}}}
    \newcommand{\NormalTok}[1]{{#1}}
    
    % Additional commands for more recent versions of Pandoc
    \newcommand{\ConstantTok}[1]{\textcolor[rgb]{0.53,0.00,0.00}{{#1}}}
    \newcommand{\SpecialCharTok}[1]{\textcolor[rgb]{0.25,0.44,0.63}{{#1}}}
    \newcommand{\VerbatimStringTok}[1]{\textcolor[rgb]{0.25,0.44,0.63}{{#1}}}
    \newcommand{\SpecialStringTok}[1]{\textcolor[rgb]{0.73,0.40,0.53}{{#1}}}
    \newcommand{\ImportTok}[1]{{#1}}
    \newcommand{\DocumentationTok}[1]{\textcolor[rgb]{0.73,0.13,0.13}{\textit{{#1}}}}
    \newcommand{\AnnotationTok}[1]{\textcolor[rgb]{0.38,0.63,0.69}{\textbf{\textit{{#1}}}}}
    \newcommand{\CommentVarTok}[1]{\textcolor[rgb]{0.38,0.63,0.69}{\textbf{\textit{{#1}}}}}
    \newcommand{\VariableTok}[1]{\textcolor[rgb]{0.10,0.09,0.49}{{#1}}}
    \newcommand{\ControlFlowTok}[1]{\textcolor[rgb]{0.00,0.44,0.13}{\textbf{{#1}}}}
    \newcommand{\OperatorTok}[1]{\textcolor[rgb]{0.40,0.40,0.40}{{#1}}}
    \newcommand{\BuiltInTok}[1]{{#1}}
    \newcommand{\ExtensionTok}[1]{{#1}}
    \newcommand{\PreprocessorTok}[1]{\textcolor[rgb]{0.74,0.48,0.00}{{#1}}}
    \newcommand{\AttributeTok}[1]{\textcolor[rgb]{0.49,0.56,0.16}{{#1}}}
    \newcommand{\InformationTok}[1]{\textcolor[rgb]{0.38,0.63,0.69}{\textbf{\textit{{#1}}}}}
    \newcommand{\WarningTok}[1]{\textcolor[rgb]{0.38,0.63,0.69}{\textbf{\textit{{#1}}}}}
    
    
    % Define a nice break command that doesn't care if a line doesn't already
    % exist.
    \def\br{\hspace*{\fill} \\* }
    % Math Jax compatability definitions
    \def\gt{>}
    \def\lt{<}
    % Document parameters
    \title{Premiere\_Cours\_Suite\_Partie1}
    
    
    

    % Pygments definitions
    
\makeatletter
\def\PY@reset{\let\PY@it=\relax \let\PY@bf=\relax%
    \let\PY@ul=\relax \let\PY@tc=\relax%
    \let\PY@bc=\relax \let\PY@ff=\relax}
\def\PY@tok#1{\csname PY@tok@#1\endcsname}
\def\PY@toks#1+{\ifx\relax#1\empty\else%
    \PY@tok{#1}\expandafter\PY@toks\fi}
\def\PY@do#1{\PY@bc{\PY@tc{\PY@ul{%
    \PY@it{\PY@bf{\PY@ff{#1}}}}}}}
\def\PY#1#2{\PY@reset\PY@toks#1+\relax+\PY@do{#2}}

\expandafter\def\csname PY@tok@w\endcsname{\def\PY@tc##1{\textcolor[rgb]{0.73,0.73,0.73}{##1}}}
\expandafter\def\csname PY@tok@c\endcsname{\let\PY@it=\textit\def\PY@tc##1{\textcolor[rgb]{0.25,0.50,0.50}{##1}}}
\expandafter\def\csname PY@tok@cp\endcsname{\def\PY@tc##1{\textcolor[rgb]{0.74,0.48,0.00}{##1}}}
\expandafter\def\csname PY@tok@k\endcsname{\let\PY@bf=\textbf\def\PY@tc##1{\textcolor[rgb]{0.00,0.50,0.00}{##1}}}
\expandafter\def\csname PY@tok@kp\endcsname{\def\PY@tc##1{\textcolor[rgb]{0.00,0.50,0.00}{##1}}}
\expandafter\def\csname PY@tok@kt\endcsname{\def\PY@tc##1{\textcolor[rgb]{0.69,0.00,0.25}{##1}}}
\expandafter\def\csname PY@tok@o\endcsname{\def\PY@tc##1{\textcolor[rgb]{0.40,0.40,0.40}{##1}}}
\expandafter\def\csname PY@tok@ow\endcsname{\let\PY@bf=\textbf\def\PY@tc##1{\textcolor[rgb]{0.67,0.13,1.00}{##1}}}
\expandafter\def\csname PY@tok@nb\endcsname{\def\PY@tc##1{\textcolor[rgb]{0.00,0.50,0.00}{##1}}}
\expandafter\def\csname PY@tok@nf\endcsname{\def\PY@tc##1{\textcolor[rgb]{0.00,0.00,1.00}{##1}}}
\expandafter\def\csname PY@tok@nc\endcsname{\let\PY@bf=\textbf\def\PY@tc##1{\textcolor[rgb]{0.00,0.00,1.00}{##1}}}
\expandafter\def\csname PY@tok@nn\endcsname{\let\PY@bf=\textbf\def\PY@tc##1{\textcolor[rgb]{0.00,0.00,1.00}{##1}}}
\expandafter\def\csname PY@tok@ne\endcsname{\let\PY@bf=\textbf\def\PY@tc##1{\textcolor[rgb]{0.82,0.25,0.23}{##1}}}
\expandafter\def\csname PY@tok@nv\endcsname{\def\PY@tc##1{\textcolor[rgb]{0.10,0.09,0.49}{##1}}}
\expandafter\def\csname PY@tok@no\endcsname{\def\PY@tc##1{\textcolor[rgb]{0.53,0.00,0.00}{##1}}}
\expandafter\def\csname PY@tok@nl\endcsname{\def\PY@tc##1{\textcolor[rgb]{0.63,0.63,0.00}{##1}}}
\expandafter\def\csname PY@tok@ni\endcsname{\let\PY@bf=\textbf\def\PY@tc##1{\textcolor[rgb]{0.60,0.60,0.60}{##1}}}
\expandafter\def\csname PY@tok@na\endcsname{\def\PY@tc##1{\textcolor[rgb]{0.49,0.56,0.16}{##1}}}
\expandafter\def\csname PY@tok@nt\endcsname{\let\PY@bf=\textbf\def\PY@tc##1{\textcolor[rgb]{0.00,0.50,0.00}{##1}}}
\expandafter\def\csname PY@tok@nd\endcsname{\def\PY@tc##1{\textcolor[rgb]{0.67,0.13,1.00}{##1}}}
\expandafter\def\csname PY@tok@s\endcsname{\def\PY@tc##1{\textcolor[rgb]{0.73,0.13,0.13}{##1}}}
\expandafter\def\csname PY@tok@sd\endcsname{\let\PY@it=\textit\def\PY@tc##1{\textcolor[rgb]{0.73,0.13,0.13}{##1}}}
\expandafter\def\csname PY@tok@si\endcsname{\let\PY@bf=\textbf\def\PY@tc##1{\textcolor[rgb]{0.73,0.40,0.53}{##1}}}
\expandafter\def\csname PY@tok@se\endcsname{\let\PY@bf=\textbf\def\PY@tc##1{\textcolor[rgb]{0.73,0.40,0.13}{##1}}}
\expandafter\def\csname PY@tok@sr\endcsname{\def\PY@tc##1{\textcolor[rgb]{0.73,0.40,0.53}{##1}}}
\expandafter\def\csname PY@tok@ss\endcsname{\def\PY@tc##1{\textcolor[rgb]{0.10,0.09,0.49}{##1}}}
\expandafter\def\csname PY@tok@sx\endcsname{\def\PY@tc##1{\textcolor[rgb]{0.00,0.50,0.00}{##1}}}
\expandafter\def\csname PY@tok@m\endcsname{\def\PY@tc##1{\textcolor[rgb]{0.40,0.40,0.40}{##1}}}
\expandafter\def\csname PY@tok@gh\endcsname{\let\PY@bf=\textbf\def\PY@tc##1{\textcolor[rgb]{0.00,0.00,0.50}{##1}}}
\expandafter\def\csname PY@tok@gu\endcsname{\let\PY@bf=\textbf\def\PY@tc##1{\textcolor[rgb]{0.50,0.00,0.50}{##1}}}
\expandafter\def\csname PY@tok@gd\endcsname{\def\PY@tc##1{\textcolor[rgb]{0.63,0.00,0.00}{##1}}}
\expandafter\def\csname PY@tok@gi\endcsname{\def\PY@tc##1{\textcolor[rgb]{0.00,0.63,0.00}{##1}}}
\expandafter\def\csname PY@tok@gr\endcsname{\def\PY@tc##1{\textcolor[rgb]{1.00,0.00,0.00}{##1}}}
\expandafter\def\csname PY@tok@ge\endcsname{\let\PY@it=\textit}
\expandafter\def\csname PY@tok@gs\endcsname{\let\PY@bf=\textbf}
\expandafter\def\csname PY@tok@gp\endcsname{\let\PY@bf=\textbf\def\PY@tc##1{\textcolor[rgb]{0.00,0.00,0.50}{##1}}}
\expandafter\def\csname PY@tok@go\endcsname{\def\PY@tc##1{\textcolor[rgb]{0.53,0.53,0.53}{##1}}}
\expandafter\def\csname PY@tok@gt\endcsname{\def\PY@tc##1{\textcolor[rgb]{0.00,0.27,0.87}{##1}}}
\expandafter\def\csname PY@tok@err\endcsname{\def\PY@bc##1{\setlength{\fboxsep}{0pt}\fcolorbox[rgb]{1.00,0.00,0.00}{1,1,1}{\strut ##1}}}
\expandafter\def\csname PY@tok@kc\endcsname{\let\PY@bf=\textbf\def\PY@tc##1{\textcolor[rgb]{0.00,0.50,0.00}{##1}}}
\expandafter\def\csname PY@tok@kd\endcsname{\let\PY@bf=\textbf\def\PY@tc##1{\textcolor[rgb]{0.00,0.50,0.00}{##1}}}
\expandafter\def\csname PY@tok@kn\endcsname{\let\PY@bf=\textbf\def\PY@tc##1{\textcolor[rgb]{0.00,0.50,0.00}{##1}}}
\expandafter\def\csname PY@tok@kr\endcsname{\let\PY@bf=\textbf\def\PY@tc##1{\textcolor[rgb]{0.00,0.50,0.00}{##1}}}
\expandafter\def\csname PY@tok@bp\endcsname{\def\PY@tc##1{\textcolor[rgb]{0.00,0.50,0.00}{##1}}}
\expandafter\def\csname PY@tok@fm\endcsname{\def\PY@tc##1{\textcolor[rgb]{0.00,0.00,1.00}{##1}}}
\expandafter\def\csname PY@tok@vc\endcsname{\def\PY@tc##1{\textcolor[rgb]{0.10,0.09,0.49}{##1}}}
\expandafter\def\csname PY@tok@vg\endcsname{\def\PY@tc##1{\textcolor[rgb]{0.10,0.09,0.49}{##1}}}
\expandafter\def\csname PY@tok@vi\endcsname{\def\PY@tc##1{\textcolor[rgb]{0.10,0.09,0.49}{##1}}}
\expandafter\def\csname PY@tok@vm\endcsname{\def\PY@tc##1{\textcolor[rgb]{0.10,0.09,0.49}{##1}}}
\expandafter\def\csname PY@tok@sa\endcsname{\def\PY@tc##1{\textcolor[rgb]{0.73,0.13,0.13}{##1}}}
\expandafter\def\csname PY@tok@sb\endcsname{\def\PY@tc##1{\textcolor[rgb]{0.73,0.13,0.13}{##1}}}
\expandafter\def\csname PY@tok@sc\endcsname{\def\PY@tc##1{\textcolor[rgb]{0.73,0.13,0.13}{##1}}}
\expandafter\def\csname PY@tok@dl\endcsname{\def\PY@tc##1{\textcolor[rgb]{0.73,0.13,0.13}{##1}}}
\expandafter\def\csname PY@tok@s2\endcsname{\def\PY@tc##1{\textcolor[rgb]{0.73,0.13,0.13}{##1}}}
\expandafter\def\csname PY@tok@sh\endcsname{\def\PY@tc##1{\textcolor[rgb]{0.73,0.13,0.13}{##1}}}
\expandafter\def\csname PY@tok@s1\endcsname{\def\PY@tc##1{\textcolor[rgb]{0.73,0.13,0.13}{##1}}}
\expandafter\def\csname PY@tok@mb\endcsname{\def\PY@tc##1{\textcolor[rgb]{0.40,0.40,0.40}{##1}}}
\expandafter\def\csname PY@tok@mf\endcsname{\def\PY@tc##1{\textcolor[rgb]{0.40,0.40,0.40}{##1}}}
\expandafter\def\csname PY@tok@mh\endcsname{\def\PY@tc##1{\textcolor[rgb]{0.40,0.40,0.40}{##1}}}
\expandafter\def\csname PY@tok@mi\endcsname{\def\PY@tc##1{\textcolor[rgb]{0.40,0.40,0.40}{##1}}}
\expandafter\def\csname PY@tok@il\endcsname{\def\PY@tc##1{\textcolor[rgb]{0.40,0.40,0.40}{##1}}}
\expandafter\def\csname PY@tok@mo\endcsname{\def\PY@tc##1{\textcolor[rgb]{0.40,0.40,0.40}{##1}}}
\expandafter\def\csname PY@tok@ch\endcsname{\let\PY@it=\textit\def\PY@tc##1{\textcolor[rgb]{0.25,0.50,0.50}{##1}}}
\expandafter\def\csname PY@tok@cm\endcsname{\let\PY@it=\textit\def\PY@tc##1{\textcolor[rgb]{0.25,0.50,0.50}{##1}}}
\expandafter\def\csname PY@tok@cpf\endcsname{\let\PY@it=\textit\def\PY@tc##1{\textcolor[rgb]{0.25,0.50,0.50}{##1}}}
\expandafter\def\csname PY@tok@c1\endcsname{\let\PY@it=\textit\def\PY@tc##1{\textcolor[rgb]{0.25,0.50,0.50}{##1}}}
\expandafter\def\csname PY@tok@cs\endcsname{\let\PY@it=\textit\def\PY@tc##1{\textcolor[rgb]{0.25,0.50,0.50}{##1}}}

\def\PYZbs{\char`\\}
\def\PYZus{\char`\_}
\def\PYZob{\char`\{}
\def\PYZcb{\char`\}}
\def\PYZca{\char`\^}
\def\PYZam{\char`\&}
\def\PYZlt{\char`\<}
\def\PYZgt{\char`\>}
\def\PYZsh{\char`\#}
\def\PYZpc{\char`\%}
\def\PYZdl{\char`\$}
\def\PYZhy{\char`\-}
\def\PYZsq{\char`\'}
\def\PYZdq{\char`\"}
\def\PYZti{\char`\~}
% for compatibility with earlier versions
\def\PYZat{@}
\def\PYZlb{[}
\def\PYZrb{]}
\makeatother


    % Exact colors from NB
    \definecolor{incolor}{rgb}{0.0, 0.0, 0.5}
    \definecolor{outcolor}{rgb}{0.545, 0.0, 0.0}



    
    % Prevent overflowing lines due to hard-to-break entities
    \sloppy 
    % Setup hyperref package
    \hypersetup{
      breaklinks=true,  % so long urls are correctly broken across lines
      colorlinks=true,
      urlcolor=urlcolor,
      linkcolor=linkcolor,
      citecolor=citecolor,
      }
    % Slightly bigger margins than the latex defaults
    
    \geometry{verbose,tmargin=1in,bmargin=1in,lmargin=1in,rmargin=1in}
    
    

    \begin{document}
    
    
    \maketitle
    
    

    
    \section{Première Cours Suites Partie
1}\label{premiuxe8re-cours-suites-partie-1}

    Sur la page
{[}https://repl.it/@fredericjunier/PremiereSuitesPartie1{]}(https://repl.it/@fredericjunier/PremiereSuitesPartie1)
vous pourrez aussi tester les codes ci-dessous.

    \subsection{Activité 1 : châteaux de
cartes}\label{activituxe9-1-chuxe2teaux-de-cartes}

    Pour tout entier naturel \(n \geqslant 1\), on a : * le nombre de cartes
au niveau \(n\) vérifie \(u_{n}=3n\) * le nombre total de cartes pour
construire \(n\) niveaux vérifie \(v_{n+1}=u_{n+1} + v_{n}\)

    \begin{Verbatim}[commandchars=\\\{\}]
{\color{incolor}In [{\color{incolor}1}]:} \PY{k}{def} \PY{n+nf}{chateau}\PY{p}{(}\PY{n}{n}\PY{p}{)}\PY{p}{:}
            \PY{n}{u} \PY{o}{=} \PY{l+m+mi}{0}
            \PY{n}{v} \PY{o}{=} \PY{l+m+mi}{0}
            \PY{k}{for} \PY{n}{k} \PY{o+ow}{in} \PY{n+nb}{range}\PY{p}{(}\PY{l+m+mi}{1}\PY{p}{,} \PY{n}{n} \PY{o}{+} \PY{l+m+mi}{1}\PY{p}{)}\PY{p}{:}
                \PY{n}{u} \PY{o}{=} \PY{n}{u} \PY{o}{+} \PY{l+m+mi}{3}  \PY{c+c1}{\PYZsh{}u = k * 3}
                \PY{n}{v} \PY{o}{=} \PY{n}{v} \PY{o}{+} \PY{n}{u}
            \PY{k}{return} \PY{n}{v}
\end{Verbatim}


    \begin{Verbatim}[commandchars=\\\{\}]
{\color{incolor}In [{\color{incolor}2}]:} \PY{p}{[}\PY{n}{chateau}\PY{p}{(}\PY{n}{n}\PY{p}{)} \PY{k}{for} \PY{n}{n} \PY{o+ow}{in} \PY{n+nb}{range}\PY{p}{(}\PY{l+m+mi}{0}\PY{p}{,} \PY{l+m+mi}{8}\PY{p}{)}\PY{p}{]}
\end{Verbatim}


\begin{Verbatim}[commandchars=\\\{\}]
{\color{outcolor}Out[{\color{outcolor}2}]:} [0, 3, 9, 18, 30, 45, 63, 84]
\end{Verbatim}
            
    \subsection{Activité 2 Modèle d'évolution d'une
population}\label{activituxe9-2-moduxe8le-duxe9volution-dune-population}

    On a \(u_{0}=27500\) étudiants en Septembre 2016.

En notant \(u_{n}\) le nombre d'étudiants en Septembre \(2016 + n\), en
juin \(2016+n+1\), après une perte de \(150\) étudiants, on a
\(u_{n}-150\) étudiants. Puis on a une augmentation de cet effectif, à
la rentrée de Septembre, cest-à-dire \(1,04(u_{n}-150)=1,04u_{n}-156\)
étudiants en Septembre \(2016+n+1\).

Pour tout entier \(n \geqslant 0\), on a donc :
\(u_{n+1}=1,04u_{n}-156\).

La capacité maximale de l'établissement est de \(33000\). D'après
l'algorithme de seuil ci-dessous, à la rentrée de Septembre \(2022\), la
capacité maximale d'accueil sera dépassée.

    \begin{Verbatim}[commandchars=\\\{\}]
{\color{incolor}In [{\color{incolor}7}]:} \PY{k}{def} \PY{n+nf}{seuil}\PY{p}{(}\PY{p}{)}\PY{p}{:}
            \PY{n}{n} \PY{o}{=} \PY{l+m+mi}{0}
            \PY{n}{u} \PY{o}{=} \PY{l+m+mi}{27500}
            \PY{k}{while} \PY{n}{u} \PY{o}{\PYZlt{}}\PY{o}{=} \PY{l+m+mi}{33000}\PY{p}{:}
                \PY{n}{n} \PY{o}{=} \PY{n}{n} \PY{o}{+} \PY{l+m+mi}{1}
                \PY{n}{u} \PY{o}{=} \PY{l+m+mf}{1.04} \PY{o}{*} \PY{n}{u} \PY{o}{\PYZhy{}} \PY{l+m+mi}{156}
            \PY{k}{return} \PY{n}{n}
\end{Verbatim}


    \subsection{Capacité 2}\label{capacituxe9-2}

    \begin{itemize}
\tightlist
\item
  Question 1 a) : \(a_{2} \approx 14400\) et \(a_{8} \approx 17200\)
\item
  Question 1 b) : Le montant de l'APA en 2013 était de \(a_{7} = 16744\)
\item
  Question 2) a) :
  \(a_{10} = a_{9} \times 1,05 = 18070 \times 1,05 = 18973,5\)
\item
  Question 2) b) :
  \href{https://lite.framacalc.org/9el4-premiereg5-lyceeparc-suites-capacite2}{feuille
  de calcul en ligne}
\end{itemize}

n

9

10

11

12

13

14

a(n)

18070

18973.5

19922.175

20918.28375

21964.1979375

23062.407834375

    \subsection{Capacité 3}\label{capacituxe9-3}

    Soit la suite définie par : \(u_0 = 4\) et, pour tout entier naturel
\(n\), \(u_{n+1} = -\frac{1}{2}u_n + 2\).

\begin{itemize}
\item
  Question 1) a) : \(u_{1}=-\frac{1}{2}u_0 + 2=0\) et
  \(u_{2}=-\frac{1}{2}0 + 2=2\)
\item
  Question 1) b) : calcul de \(u_{n}\) avec une fonction Python
\end{itemize}

\begin{Shaded}
\begin{Highlighting}[]
\KeywordTok{def}\NormalTok{ suiteU_capacite3_question1(n):}
\NormalTok{    u }\OperatorTok{=} \DecValTok{4}
    \ControlFlowTok{for}\NormalTok{ k }\KeywordTok{in} \BuiltInTok{range}\NormalTok{(n):}
\NormalTok{        u }\OperatorTok{=} \FloatTok{-0.5} \OperatorTok{*}\NormalTok{ u }\OperatorTok{+} \DecValTok{2}
    \ControlFlowTok{return}\NormalTok{ u}
\end{Highlighting}
\end{Shaded}

    \begin{Verbatim}[commandchars=\\\{\}]
{\color{incolor}In [{\color{incolor}2}]:} \PY{k}{def} \PY{n+nf}{suiteU\PYZus{}capacite3\PYZus{}question1}\PY{p}{(}\PY{n}{n}\PY{p}{)}\PY{p}{:}
            \PY{n}{u} \PY{o}{=} \PY{l+m+mi}{4}
            \PY{k}{for} \PY{n}{k} \PY{o+ow}{in} \PY{n+nb}{range}\PY{p}{(}\PY{n}{n}\PY{p}{)}\PY{p}{:}
                \PY{n}{u} \PY{o}{=} \PY{o}{\PYZhy{}}\PY{l+m+mf}{0.5} \PY{o}{*} \PY{n}{u} \PY{o}{+} \PY{l+m+mi}{2}
            \PY{k}{return} \PY{n}{u}
        
        \PY{c+c1}{\PYZsh{}calcul de u(14)}
        \PY{n+nb}{print}\PY{p}{(}\PY{n}{suiteU\PYZus{}capacite3\PYZus{}question1}\PY{p}{(}\PY{l+m+mi}{14}\PY{p}{)}\PY{p}{)}
\end{Verbatim}


    \begin{Verbatim}[commandchars=\\\{\}]
1.33349609375

    \end{Verbatim}

    Soit la suite définie par \(u_{0}=2\) et pour tout entier naturel \(n\),
par \(u_{n+1}=u_{n}+n^2+1\).

\begin{itemize}
\item
  Question 1) a) : \(u_{1}=u_0 + 0^{}=0\) et
  \(u_{2}=-\frac{1}{2}\times 0 + 2=2\)
\item
  Question 1) b) : calcul de \(u_{n}\) avec une fonction Python
\end{itemize}

\begin{Shaded}
\begin{Highlighting}[]
\KeywordTok{def}\NormalTok{ suiteU_capacite3_question2(n):}
\NormalTok{    u }\OperatorTok{=} \DecValTok{2}
    \ControlFlowTok{for}\NormalTok{ k }\KeywordTok{in} \BuiltInTok{range}\NormalTok{(n):}
\NormalTok{        u }\OperatorTok{=}\NormalTok{ u }\OperatorTok{+}\NormalTok{ k }\OperatorTok{**} \DecValTok{2} \OperatorTok{+} \DecValTok{1}
    \ControlFlowTok{return}\NormalTok{ u}
\end{Highlighting}
\end{Shaded}

    \begin{Verbatim}[commandchars=\\\{\}]
{\color{incolor}In [{\color{incolor}3}]:} \PY{k}{def} \PY{n+nf}{suiteU\PYZus{}capacite3\PYZus{}question2}\PY{p}{(}\PY{n}{n}\PY{p}{)}\PY{p}{:}
            \PY{n}{u} \PY{o}{=} \PY{l+m+mi}{2}
            \PY{k}{for} \PY{n}{k} \PY{o+ow}{in} \PY{n+nb}{range}\PY{p}{(}\PY{n}{n}\PY{p}{)}\PY{p}{:}
                \PY{n}{u} \PY{o}{=} \PY{n}{u} \PY{o}{+} \PY{n}{k} \PY{o}{*}\PY{o}{*} \PY{l+m+mi}{2} \PY{o}{+} \PY{l+m+mi}{1}
            \PY{k}{return} \PY{n}{u}
        
        \PY{n+nb}{print}\PY{p}{(}\PY{n}{suiteU\PYZus{}capacite3\PYZus{}question2}\PY{p}{(}\PY{l+m+mi}{10}\PY{p}{)}\PY{p}{)}
\end{Verbatim}


    \begin{Verbatim}[commandchars=\\\{\}]
297

    \end{Verbatim}

    \subsection{Capacité 4}\label{capacituxe9-4}

\begin{itemize}
\item
  Question 1) : soit la suite définie pour tout entier \(n \geqslant 1\)
  par \(v_{n}=\frac{2^{n}+1}{2+(-1)^{n}2^{n+1}}\).

  \begin{itemize}
  \tightlist
  \item
    \(v_{1}=\frac{2^{1}+1}{2+(-1)^{1}2^{1+1}}=-\frac{3}{4}\)
  \item
    \(v_{2}=\frac{2^{2}+1}{2+(-1)^{2}2^{2+1}}=\frac{5}{10}\)
  \item
    \(v_{3}=\frac{2^{3}+1}{2+(-1)^{3}2^{3+1}}=-\frac{9}{14}\)
  \end{itemize}
\item
  Question 2) : soit la suite définie par \(u_{0}=0\) et pour tout
  entier \(n \geqslant 1\), \(u_{n}=u_{n-1} + 2n - 1\)

  \begin{itemize}
  \tightlist
  \item
    \(u_{1}=u_{0} + 2\times 1 - 1 = 1\)
  \item
    \(u_{2}=u_{1} + 2\times 2 - 1 = 4\)
  \item
    \(u_{3}=u_{2} + 2\times 3 - 1 = 9\)
  \end{itemize}
\end{itemize}

Calculs de tous les termes entre \(u_{0}\) et \(u_{n}\) avec une
fonction Python

\begin{Shaded}
\begin{Highlighting}[]
\KeywordTok{def}\NormalTok{ suiteU_capacite4(n):}
\NormalTok{    u }\OperatorTok{=} \DecValTok{0}
    \BuiltInTok{print}\NormalTok{(u)}
    \ControlFlowTok{for}\NormalTok{ k }\KeywordTok{in} \BuiltInTok{range}\NormalTok{(}\DecValTok{1}\NormalTok{, n }\OperatorTok{+} \DecValTok{1}\NormalTok{):}
\NormalTok{        u }\OperatorTok{=}\NormalTok{ u }\OperatorTok{+} \DecValTok{2} \OperatorTok{*}\NormalTok{ k }\OperatorTok{-} \DecValTok{1}
        \BuiltInTok{print}\NormalTok{(u)}
\end{Highlighting}
\end{Shaded}

On peut conjecturer que pour tout entier \(n \geqslant 0\), on a
\(u_{n}=n^2\).

    \begin{Verbatim}[commandchars=\\\{\}]
{\color{incolor}In [{\color{incolor}4}]:} \PY{k}{def} \PY{n+nf}{suiteU\PYZus{}capacite4}\PY{p}{(}\PY{n}{n}\PY{p}{)}\PY{p}{:}
            \PY{n}{u} \PY{o}{=} \PY{l+m+mi}{0}
            \PY{n+nb}{print}\PY{p}{(}\PY{n}{u}\PY{p}{)}
            \PY{k}{for} \PY{n}{k} \PY{o+ow}{in} \PY{n+nb}{range}\PY{p}{(}\PY{l+m+mi}{1}\PY{p}{,} \PY{n}{n} \PY{o}{+} \PY{l+m+mi}{1}\PY{p}{)}\PY{p}{:}
                \PY{n}{u} \PY{o}{=} \PY{n}{u} \PY{o}{+} \PY{l+m+mi}{2} \PY{o}{*} \PY{n}{k} \PY{o}{\PYZhy{}} \PY{l+m+mi}{1}
                \PY{n+nb}{print}\PY{p}{(}\PY{n}{u}\PY{p}{)}
        
        \PY{n}{suiteU\PYZus{}capacite4}\PY{p}{(}\PY{l+m+mi}{20}\PY{p}{)}
\end{Verbatim}


    \begin{Verbatim}[commandchars=\\\{\}]
0
1
4
9
16
25
36
49
64
81
100
121
144
169
196
225
256
289
324
361
400

    \end{Verbatim}

\begin{Verbatim}[commandchars=\\\{\}]
{\color{outcolor}Out[{\color{outcolor}4}]:} 400
\end{Verbatim}
            
    \subsection{Capacité 5 : manipuler les listes en
Python}\label{capacituxe9-5-manipuler-les-listes-en-python}

    \begin{Verbatim}[commandchars=\\\{\}]
{\color{incolor}In [{\color{incolor}9}]:} \PY{c+c1}{\PYZsh{}\PYZsh{}\PYZsh{}Question 1}
        \PY{n}{L1} \PY{o}{=} \PY{p}{[}\PY{l+m+mi}{852}\PY{p}{,} \PY{l+m+mi}{843}\PY{p}{,} \PY{l+m+mi}{954}\PY{p}{]}
        \PY{n+nb}{print}\PY{p}{(}\PY{n}{L1}\PY{p}{[}\PY{l+m+mi}{1}\PY{p}{]}\PY{p}{)}
        \PY{c+c1}{\PYZsh{}843}
        \PY{n+nb}{print}\PY{p}{(}\PY{n}{L1}\PY{p}{[}\PY{l+m+mi}{0}\PY{p}{]}\PY{p}{)}
        \PY{c+c1}{\PYZsh{}852}
        \PY{c+c1}{\PYZsh{}print(L[3])}
        \PY{c+c1}{\PYZsh{}provoque une erreur}
        
        \PY{c+c1}{\PYZsh{}\PYZsh{}\PYZsh{}Question 2}
        \PY{n}{L2} \PY{o}{=} \PY{p}{[}\PY{n}{k} \PY{o}{*} \PY{l+m+mi}{2} \PY{o}{\PYZhy{}} \PY{l+m+mi}{1} \PY{k}{for} \PY{n}{k} \PY{o+ow}{in} \PY{n+nb}{range}\PY{p}{(}\PY{l+m+mi}{3}\PY{p}{)}\PY{p}{]}
        
        
        \PY{c+c1}{\PYZsh{}\PYZsh{}\PYZsh{}Question 3}
        \PY{k+kn}{from} \PY{n+nn}{math} \PY{k}{import} \PY{n}{sin}
        \PY{n}{L3} \PY{o}{=} \PY{p}{[}\PY{p}{]}
        \PY{k}{for} \PY{n}{k} \PY{o+ow}{in} \PY{n+nb}{range}\PY{p}{(}\PY{l+m+mi}{1}\PY{p}{,} \PY{l+m+mi}{50}\PY{p}{)}\PY{p}{:}
            \PY{k}{if} \PY{n}{sin}\PY{p}{(}\PY{n}{k}\PY{p}{)} \PY{o}{\PYZgt{}}\PY{o}{=} \PY{l+m+mi}{0}\PY{p}{:}
                \PY{n}{L3}\PY{o}{.}\PY{n}{append}\PY{p}{(}\PY{n}{k}\PY{p}{)}
        \PY{c+c1}{\PYZsh{}équivalent à }
        \PY{n}{L32} \PY{o}{=} \PY{p}{[}\PY{n}{k} \PY{k}{for} \PY{n}{k} \PY{o+ow}{in} \PY{n+nb}{range}\PY{p}{(}\PY{l+m+mi}{1}\PY{p}{,} \PY{l+m+mi}{50}\PY{p}{)} \PY{k}{if} \PY{n}{sin}\PY{p}{(}\PY{n}{k}\PY{p}{)} \PY{o}{\PYZgt{}}\PY{o}{=} \PY{l+m+mi}{0}\PY{p}{]}
        \PY{n+nb}{print}\PY{p}{(}\PY{n}{L3} \PY{o}{==} \PY{n}{L32}\PY{p}{)}
        \PY{c+c1}{\PYZsh{}\PYZsh{}True}
        
        \PY{c+c1}{\PYZsh{}\PYZsh{}\PYZsh{} Question 4}
        \PY{n}{L4} \PY{o}{=} \PY{n+nb}{list}\PY{p}{(}\PY{n+nb}{range}\PY{p}{(}\PY{l+m+mi}{2}\PY{p}{,} \PY{l+m+mi}{5}\PY{p}{)}\PY{p}{)}
        \PY{n}{L4}\PY{o}{.}\PY{n}{pop}\PY{p}{(}\PY{p}{)}
        \PY{n}{L4}\PY{o}{.}\PY{n}{append}\PY{p}{(}\PY{l+m+mi}{14}\PY{p}{)}
        \PY{n}{L4}\PY{o}{.}\PY{n}{pop}\PY{p}{(}\PY{l+m+mi}{1}\PY{p}{)}
        \PY{n}{L4}\PY{o}{.}\PY{n}{pop}\PY{p}{(}\PY{l+m+mi}{1}\PY{p}{)}
        \PY{n}{L4}\PY{o}{.}\PY{n}{append}\PY{p}{(}\PY{l+m+mi}{16}\PY{p}{)}
        \PY{n+nb}{print}\PY{p}{(}\PY{n}{L4}\PY{p}{)}
\end{Verbatim}


    \begin{Verbatim}[commandchars=\\\{\}]
843
852
True
[2, 16]

    \end{Verbatim}

    \subsection{Suite de syracuse}\label{suite-de-syracuse}

    \begin{Verbatim}[commandchars=\\\{\}]
{\color{incolor}In [{\color{incolor}1}]:} \PY{k}{def} \PY{n+nf}{syracuse}\PY{p}{(}\PY{n}{u} \PY{p}{,} \PY{n}{n}\PY{p}{)}\PY{p}{:}
            \PY{l+s+sd}{\PYZdq{}\PYZdq{}\PYZdq{}Retourne la liste des premiers termes }
        \PY{l+s+sd}{    de la suite de syracuse de premier terme u\PYZdq{}\PYZdq{}\PYZdq{}}
            \PY{n}{L} \PY{o}{=} \PY{p}{[}\PY{n}{u}\PY{p}{]}
            \PY{k}{for} \PY{n}{k} \PY{o+ow}{in} \PY{n+nb}{range}\PY{p}{(}\PY{n}{n}\PY{p}{)}\PY{p}{:}
                \PY{k}{if} \PY{n}{u} \PY{o}{\PYZpc{}} \PY{l+m+mi}{2} \PY{o}{==} \PY{l+m+mi}{0}\PY{p}{:}
                    \PY{n}{u} \PY{o}{=} \PY{n}{u} \PY{o}{/}\PY{o}{/} \PY{l+m+mi}{2}
                \PY{k}{else}\PY{p}{:}
                    \PY{n}{u} \PY{o}{=} \PY{l+m+mi}{3} \PY{o}{*} \PY{n}{u} \PY{o}{+} \PY{l+m+mi}{1}
                \PY{n}{L}\PY{o}{.}\PY{n}{append}\PY{p}{(}\PY{n}{u}\PY{p}{)}
            \PY{k}{return} \PY{n}{L}
\end{Verbatim}


    \begin{Verbatim}[commandchars=\\\{\}]
{\color{incolor}In [{\color{incolor}4}]:} \PY{n}{syracuse}\PY{p}{(}\PY{l+m+mi}{634} \PY{p}{,} \PY{l+m+mi}{40}\PY{p}{)}
\end{Verbatim}


\begin{Verbatim}[commandchars=\\\{\}]
{\color{outcolor}Out[{\color{outcolor}4}]:} [634,
         317,
         952,
         476,
         238,
         119,
         358,
         179,
         538,
         269,
         808,
         404,
         202,
         101,
         304,
         152,
         76,
         38,
         19,
         58,
         29,
         88,
         44,
         22,
         11,
         34,
         17,
         52,
         26,
         13,
         40,
         20,
         10,
         5,
         16,
         8,
         4,
         2,
         1,
         4,
         2]
\end{Verbatim}
            
    \begin{Verbatim}[commandchars=\\\{\}]
{\color{incolor}In [{\color{incolor}7}]:} \PY{k}{for} \PY{n}{k} \PY{o+ow}{in} \PY{n+nb}{range}\PY{p}{(}\PY{l+m+mi}{10}\PY{p}{,} \PY{l+m+mi}{21}\PY{p}{)}\PY{p}{:}
            \PY{n+nb}{print}\PY{p}{(}\PY{n}{syracuse}\PY{p}{(}\PY{n}{k}\PY{p}{,} \PY{l+m+mi}{21}\PY{p}{)}\PY{p}{)}
\end{Verbatim}


    \begin{Verbatim}[commandchars=\\\{\}]
[10, 5, 16, 8, 4, 2, 1, 4, 2, 1, 4, 2, 1, 4, 2, 1, 4, 2, 1, 4, 2, 1]
[11, 34, 17, 52, 26, 13, 40, 20, 10, 5, 16, 8, 4, 2, 1, 4, 2, 1, 4, 2, 1, 4]
[12, 6, 3, 10, 5, 16, 8, 4, 2, 1, 4, 2, 1, 4, 2, 1, 4, 2, 1, 4, 2, 1]
[13, 40, 20, 10, 5, 16, 8, 4, 2, 1, 4, 2, 1, 4, 2, 1, 4, 2, 1, 4, 2, 1]
[14, 7, 22, 11, 34, 17, 52, 26, 13, 40, 20, 10, 5, 16, 8, 4, 2, 1, 4, 2, 1, 4]
[15, 46, 23, 70, 35, 106, 53, 160, 80, 40, 20, 10, 5, 16, 8, 4, 2, 1, 4, 2, 1, 4]
[16, 8, 4, 2, 1, 4, 2, 1, 4, 2, 1, 4, 2, 1, 4, 2, 1, 4, 2, 1, 4, 2]
[17, 52, 26, 13, 40, 20, 10, 5, 16, 8, 4, 2, 1, 4, 2, 1, 4, 2, 1, 4, 2, 1]
[18, 9, 28, 14, 7, 22, 11, 34, 17, 52, 26, 13, 40, 20, 10, 5, 16, 8, 4, 2, 1, 4]
[19, 58, 29, 88, 44, 22, 11, 34, 17, 52, 26, 13, 40, 20, 10, 5, 16, 8, 4, 2, 1, 4]
[20, 10, 5, 16, 8, 4, 2, 1, 4, 2, 1, 4, 2, 1, 4, 2, 1, 4, 2, 1, 4, 2]

    \end{Verbatim}

    \begin{Verbatim}[commandchars=\\\{\}]
{\color{incolor}In [{\color{incolor}8}]:} \PY{k}{def} \PY{n+nf}{tempsVol}\PY{p}{(}\PY{n}{u}\PY{p}{)}\PY{p}{:}
            \PY{l+s+sd}{\PYZdq{}\PYZdq{}\PYZdq{}Retourne le plus petit indice du terme de la suite de syracuse}
        \PY{l+s+sd}{    de premier terme u, qui est égal à 1\PYZdq{}\PYZdq{}\PYZdq{}}
            \PY{n}{i} \PY{o}{=} \PY{l+m+mi}{0}
            \PY{k}{while} \PY{n}{u} \PY{o}{!=} \PY{l+m+mi}{1}\PY{p}{:}
                \PY{k}{if} \PY{n}{u} \PY{o}{\PYZpc{}} \PY{l+m+mi}{2} \PY{o}{==} \PY{l+m+mi}{0}\PY{p}{:}
                    \PY{n}{u} \PY{o}{=} \PY{n}{u} \PY{o}{/}\PY{o}{/} \PY{l+m+mi}{2}
                \PY{k}{else}\PY{p}{:}
                    \PY{n}{u} \PY{o}{=} \PY{l+m+mi}{3} \PY{o}{*} \PY{n}{u} \PY{o}{+} \PY{l+m+mi}{1}
                \PY{n}{i} \PY{o}{=} \PY{n}{i} \PY{o}{+} \PY{l+m+mi}{1}
            \PY{k}{return} \PY{n}{i}
\end{Verbatim}


    \begin{Verbatim}[commandchars=\\\{\}]
{\color{incolor}In [{\color{incolor}9}]:} \PY{n}{tempsVol}\PY{p}{(}\PY{l+m+mi}{634}\PY{p}{)}
\end{Verbatim}


\begin{Verbatim}[commandchars=\\\{\}]
{\color{outcolor}Out[{\color{outcolor}9}]:} 38
\end{Verbatim}
            
    \subsection{Capacité 6 : Suite définie par des motifs géométriques ou
combinatoires}\label{capacituxe9-6-suite-duxe9finie-par-des-motifs-guxe9omuxe9triques-ou-combinatoires}

\begin{itemize}
\item
  Question 1) : On a \(t_{1}=1\) et pour tout entier \(n \geqslant 2\),
  on a \(t_{n}=t_{n-1}+n\).
\item
  Question 2) : On admet que pour tout entier \(n \geqslant 1\), on a
  \(t_{n}+t_{n-1}=n^{2}\). On en déduit que \(t_{n}+t_{n}-n=n^{2}\)
  c'est-à-dire \(t_{n}=\frac{n(n+1)}{2}\). Notons que pour tout entier
  \(n \geqslant 1\), \(t_{n}=1+2+\ldots +n\) donc
  \(1+2+\ldots +n==\frac{n(n+1)}{2}\)
\item
  Question 3) a) : On a \(u_{1}=0\) et pour tout entier
  \(n \geqslant 1\), on a \(u_{n+1}=u_{n}+n\) ou encore pour tout entier
  \(n \geqslant 2\), on a \(u_{n}=u_{n-1}+n-1\).
\item
  Question 3) b) : On peut remarquer que \(u_{n}=1+2+\ldots+n-1\).
  D'après la question précédente, on a, pour tout entier
  \(n \geqslant 1\), \(u_{n}=\frac{(n-1)n}{2}\). On peut aussi noter que
  \(u_{1}=0=t_{1}-1\) puis \(u_{2}=u_{1}-1+1=t_{1}\) puis
  \(u_{3}=t_{1}+2=t_{2}\) puis \(u_{4}=t_{2}+3=t_{3}\). En terminale, on
  pourra démontrer par récurrence que pour tout entier
  \(n \geqslant 2\), on a \(u_{n}=t_{n-1}=\frac{(n-1)n}{2}\).
\item
  Question 4) : On peut assimiler le nombre total de poignées de mains
  échangées dans une assemblée de \(n\) personnes qui se saluent toutes
  deux à deux, par la valeur de \(u_{n}\) définie dans la question
  précédente. En effet, on peut considérer que les personnes arrivent
  successivement dans la salle et que tout nouvel arrivant salue toutes
  les personnes déjà présentes.
\end{itemize}

On résout donc l'équation
\(\frac{(n-1)n}{2}=45 \Longleftrightarrow n^{2}-n-90=0 \Longleftrightarrow \begin{cases}n=\frac{1+19}{2}=10 \\ \text{ou }\frac{1-19}{2}=-9<0 \text{ impossible} \end{cases}\).

    \subsection{Algorithmique 2 : Factorielle de
n}\label{algorithmique-2-factorielle-de-n}

    \begin{Verbatim}[commandchars=\\\{\}]
{\color{incolor}In [{\color{incolor}10}]:} \PY{k}{def} \PY{n+nf}{factorielle}\PY{p}{(}\PY{n}{n}\PY{p}{)}\PY{p}{:}
             \PY{n}{u} \PY{o}{=} \PY{l+m+mi}{1}
             \PY{k}{for} \PY{n}{k} \PY{o+ow}{in} \PY{n+nb}{range}\PY{p}{(}\PY{l+m+mi}{1}\PY{p}{,} \PY{n}{n} \PY{o}{+} \PY{l+m+mi}{1}\PY{p}{)}\PY{p}{:}
                 \PY{n}{u} \PY{o}{=} \PY{n}{u} \PY{o}{*} \PY{n}{k}
             \PY{k}{return} \PY{n}{u}
\end{Verbatim}


    \begin{Verbatim}[commandchars=\\\{\}]
{\color{incolor}In [{\color{incolor}11}]:} \PY{p}{[}\PY{n}{factorielle}\PY{p}{(}\PY{n}{n}\PY{p}{)} \PY{k}{for} \PY{n}{n} \PY{o+ow}{in} \PY{n+nb}{range}\PY{p}{(}\PY{l+m+mi}{10}\PY{p}{)}\PY{p}{]}
\end{Verbatim}


\begin{Verbatim}[commandchars=\\\{\}]
{\color{outcolor}Out[{\color{outcolor}11}]:} [1, 1, 2, 6, 24, 120, 720, 5040, 40320, 362880]
\end{Verbatim}
            
    \subsection{Capacité 7 : Modéliser un phénomène discret à croissance
linéaire}\label{capacituxe9-7-moduxe9liser-un-phuxe9nomuxe8ne-discret-uxe0-croissance-linuxe9aire}

\begin{itemize}
\tightlist
\item
  Question 1) : \(v_{1}=20\) et \(v_{2}=v_{1}-0,6=19,4\).
\item
  Question 2) : pour tout entier \(n\) tel que
  \(1 \leqslant n \leqslant 23\), on a \(v_{n+1}=v_{n}-0,6\).
\item
  Question 3) : au douzième mois, le montant de la mensualité est de
  \(v_{12}=v_{1}-11 \times 0,6=20-6,6=13,4\) euros. Plus généralement,
  pour tout entier \(n\) tel que \(1 \leqslant n \leqslant 24\), on aura
  \(v_{n}=v_{1}-0,6(n-1)=20,6-0,6n\).
\end{itemize}

    \subsection{Capacité 8 : Modéliser un phénomène discret à croissance
exponentielle}\label{capacituxe9-8-moduxe9liser-un-phuxe9nomuxe8ne-discret-uxe0-croissance-exponentielle}

\begin{itemize}
\tightlist
\item
  Question 1) : \(C_{1}=820\) euros et
  \(C_{2}=C_{1} \times 1,025=840,5\) euros.
\item
  Question 2) : Pour tout entier naturel \(n\), on a
  \(C_{n+1}=1,025 \times C_{n}\) (formule de récurrence) et
  \(C_{n}=C_{0} \times 1,025^{n}\) (formule explicite)
\item
  Question 3) : Algorithme de seuil en Python, fonction retournant le
  plus petit entier \(n\) tel que \(C_{n}>s\)
\end{itemize}

\begin{Shaded}
\begin{Highlighting}[]
\KeywordTok{def}\NormalTok{ seuil(s):}
\NormalTok{    c }\OperatorTok{=} \DecValTok{800}
\NormalTok{    n }\OperatorTok{=} \DecValTok{0}
    \ControlFlowTok{while}\NormalTok{ c }\OperatorTok{<=}\NormalTok{ s:}
\NormalTok{        c }\OperatorTok{=} \FloatTok{1.025} \OperatorTok{*}\NormalTok{ c}
\NormalTok{        n }\OperatorTok{=}\NormalTok{n }\OperatorTok{+} \DecValTok{1}
    \ControlFlowTok{return}\NormalTok{ n}
\end{Highlighting}
\end{Shaded}

    \begin{Verbatim}[commandchars=\\\{\}]
{\color{incolor}In [{\color{incolor}13}]:} \PY{k}{def} \PY{n+nf}{seuil}\PY{p}{(}\PY{n}{s}\PY{p}{)}\PY{p}{:}
             \PY{n}{c} \PY{o}{=} \PY{l+m+mi}{800}
             \PY{n}{n} \PY{o}{=} \PY{l+m+mi}{0}
             \PY{n+nb}{print}\PY{p}{(}\PY{n}{n}\PY{p}{,} \PY{n}{c}\PY{p}{)}
             \PY{k}{while} \PY{n}{c} \PY{o}{\PYZlt{}}\PY{o}{=} \PY{n}{s}\PY{p}{:}
                 \PY{n}{c} \PY{o}{=} \PY{l+m+mf}{1.025} \PY{o}{*} \PY{n}{c}
                 \PY{n}{n} \PY{o}{=}\PY{n}{n} \PY{o}{+} \PY{l+m+mi}{1}
                 \PY{n+nb}{print}\PY{p}{(}\PY{n}{n}\PY{p}{,} \PY{n}{c}\PY{p}{)}
             \PY{k}{return} \PY{n}{n}
         
         \PY{n+nb}{print}\PY{p}{(}\PY{n}{seuil}\PY{p}{(}\PY{l+m+mi}{1000}\PY{p}{)}\PY{p}{)}
\end{Verbatim}


    \begin{Verbatim}[commandchars=\\\{\}]
0 800
1 819.9999999999999
2 840.4999999999998
3 861.5124999999997
4 883.0503124999997
5 905.1265703124996
6 927.7547345703119
7 950.9486029345696
8 974.7223180079338
9 999.0903759581321
10 1024.0676353570852
10

    \end{Verbatim}

    \subsection{Capacité 11 Déterminer une relation pour une suite définie
par un motif
géométrique}\label{capacituxe9-11-duxe9terminer-une-relation-pour-une-suite-duxe9finie-par-un-motif-guxe9omuxe9trique}

    \begin{itemize}
\item
  \textbf{Question 1} : pour tout entier \(n \geqslant 1\), on a
  \(a_{n+1}=a_{n}=2\) avec \(a_{1}=1\). La suite \(\left(a_{n}\right)\)
  est donc arithmétique de raison \(2\) et pour tout entier naturel
  \(n\geqslant 1\), on a \(a_{n}=a_{1} + (n-1) \times 2=1+2(n-1)=2n-1\).
\item
  \textbf{Question 2} : D'après la formule sur la somme des termes
  consécutifs d'uen suite arithmétique, on a
  \(\sum_{k=1}^{n}a_{k}=\frac{a_{1}+a_{n}}{2} \times n= \frac{1+2n-1}{2} \times n = n^{2}\).
\item
  \textbf{Question 3} : On résout l'inéquation
  \(\sum_{k=1}^{n}a_{k} \geqslant 1 \Longleftrightarrow n^{2} \geqslant 1000 \Longleftrightarrow n \geqslant \sqrt{1000} \Longleftrightarrow n \geqslant 32\).
  Le robot doit donc effectuer au moin \(32\) trajets en ligne droite
  (le dernier sera incomplet) et \(31\) virage, ce qui pour une distance
  de \(1\) kilomètres soit \(10^{5}\) centimètres représente un temps de
  \(\frac{10^{5}}{20} + 31 \times 2 =5062\) secondes.
\end{itemize}

    \subsection{Capacité 12 Modéliser un phénomène discret à croissance
exponentielle, calculer le terme général d'une suite
géométrique}\label{capacituxe9-12-moduxe9liser-un-phuxe9nomuxe8ne-discret-uxe0-croissance-exponentielle-calculer-le-terme-guxe9nuxe9ral-dune-suite-guxe9omuxe9trique}

    \begin{itemize}
\tightlist
\item
  \textbf{Question 1} : \(u_{0}=60\), \(u_{1}=30\) et \(u_{2}=15\).
\item
  \textbf{Question 2} : Pour tout entier naturel \(n\), on a
  \(u_{n+1}=0,5u_{n}\) donc la suite \(\left(u_{n}\right)\) est
  géométrique de raison \(0,5\) et donc
  \(u_{n}=0,5^{n}u_{0}=\frac{60}{2^{n}}\). On en déduit que
  \(u_{5}=\frac{60}{2^{5}}=1,875\).
\item
  \textbf{Question 3} : Fonction \texttt{seuil()} qui retourne le plus
  petit entier n à partir duquel \(u_{n}<0,25\). Elle retourne \(8\)
  donc au bout de \(8\) demi-vies soit \(8 \times 3=24\) jours,
  l'activité radioactive de cet échantillon est strictement inférieure à
  \(0,25\).
\end{itemize}

    \begin{Verbatim}[commandchars=\\\{\}]
{\color{incolor}In [{\color{incolor}2}]:} \PY{k}{def} \PY{n+nf}{seuil}\PY{p}{(}\PY{p}{)}\PY{p}{:}
            \PY{n}{u} \PY{o}{=} \PY{l+m+mi}{60} 
            \PY{n}{n} \PY{o}{=} \PY{l+m+mi}{0}
            \PY{k}{while} \PY{n}{u} \PY{o}{\PYZgt{}}\PY{o}{=} \PY{l+m+mf}{0.25}\PY{p}{:}
                \PY{n}{u} \PY{o}{=} \PY{n}{u} \PY{o}{/} \PY{l+m+mi}{2}
                \PY{n}{n} \PY{o}{=} \PY{n}{n} \PY{o}{+} \PY{l+m+mi}{1}
            \PY{k}{return} \PY{n}{n}
\end{Verbatim}


    \begin{Verbatim}[commandchars=\\\{\}]
{\color{incolor}In [{\color{incolor}3}]:} \PY{n}{seuil}\PY{p}{(}\PY{p}{)}
\end{Verbatim}


\begin{Verbatim}[commandchars=\\\{\}]
{\color{outcolor}Out[{\color{outcolor}3}]:} 8
\end{Verbatim}
            

    % Add a bibliography block to the postdoc
    
    
    
    \end{document}
