\documentclass[11pt, hyperref={urlcolor=red,% Liens vers une page web
            linkcolor=blue, %Liens internes au document
            colorlinks=true}]{beamer}  
            
\usetheme{Warsaw} 

%thèmes prédéfinis de Beamer 
%Antibes, boxes, classic, Darmstadt, Madrid
% Montpellier, Warsaw, Bergen, Berkeley, Goettingen, sidebar


%%%%%%%Tit\title{Titre principal}

\title[exponentielle]{Exemples du cours sur l'exponentielle 2019/2020}
%\subtitle{Sous titre}
\author[F.Junier]{Fr\'ed\'eric Junier}
\institute[Le Parc]{{\centering Lyc\'ee du Parc \\
1 Boulevard Anatole France \\ 69006 Lyon }}
\date[\today]{\today}

\usepackage{etex}

%%%%%%%%%%%%Encodage du fichier source %%%%%%%%%%%
\usepackage[T1]{fontenc}
\usepackage[utf8]{inputenc}

\usepackage{lmodern}
\usepackage{url}
\usepackage[np]{numprint}


%%%%%%%%%%%%Là encore il y a de grosses différences entre le monde anglo-saxon et les francophones.Le séparateur des décimales est un point en anglais et une virgule en français. Leséparateur des milliers est une virgule en anglais et une espace insécable en français. Ilest préférable d’utiliser le package numprint (\usepackage{numprint}) qui associé àfrenchb produira la bonne typographie.
%123456789 = 123456789 \numprint{123456789} = 123 456 789  \numprint{3,1415926535897932384626} = 3,141 592 653 589 793 238 462 6  \numprint{12.34} = 12,34  En plus tu peux préciser les unités de cette façon : \numprint[kg]{12.34} = 12,34 kg ou encore \numprint[\degres C]{22} = 22°C Si tu veux utiliser le raccourci \np{} au lieu de \numprint{}, il te faut charger le package de cette façon : \usepackage[np]{numprint}


%%%%%%%%%%PSTricks%%%%%%%%%%%%

\usepackage{pstricks,pst-plot,pst-text,pst-tree,pst-eps,pst-fill,pst-node,pst-math,pstricks-add,pst-xkey,pst-eucl}


%%%%%%%Tikz%%%%%%%%%%%%%%%
\usepackage{pgf,tikz,tkz-tab}
% Pour les tableaux de signes ou de variations avec tkz-tab voir https://zestedesavoir.com/tutoriels/439/des-tableaux-de-variations-et-de-signes-avec-latex/#1-13389_tikz-un-package-qui-en-a-dans-le-ventre
\usetikzlibrary{arrows}
\usetikzlibrary{shapes.geometric}
\usetikzlibrary{shapes.geometric}
\usetikzlibrary{petri}
\usetikzlibrary{decorations}
\usetikzlibrary{arrows}
\usetikzlibrary{math}
 %Variables must be declared in a tikzmath environment but
       % can be used outside
%       \tikzmath{int \n; \n = 508; \x1 = 1; \y1 =1; 
%                   %computations are also possible
%                    \x2 = \x1 + 1; \y2 =\y1 +3; } 


%%%%%%%%%%%%%%%%%%%%%%%%%%%%%%%%%%%%%%%%
%%%%%%%%%%%Commandes Tikz Perso%%%%%%%%%%%%%%%

% Définition des nouvelles options xmin, xmax, ymin, ymax
% Valeurs par défaut : -3, 3, -3, 3
\tikzset{
xmin/.store in=\xmin, xmin/.default=-3, xmin=-3,
xmax/.store in=\xmax, xmax/.default=3, xmax=3,
ymin/.store in=\ymin, ymin/.default=-3, ymin=-3,
ymax/.store in=\ymax, ymax/.default=3, ymax=3,
}
% Commande qui trace la grille entre (xmin,ymin) et (xmax,ymax)
\newcommand {\grille}[2]
{\draw[help lines,black, thick] (\xmin,\ymin) grid[xstep=#1, ystep=#2] (\xmax,\ymax);}
% Commande \axes
\newcommand {\axes} {
\draw[->,very thick] (\xmin,0) -- (\xmax,0);
\draw[->,very thick] (0,\ymin) -- (0,\ymax);
\draw (0.95*\xmax, 0) node[above] {$x$};
\draw (0, 0.95*\ymax) node[left] {$y$};
}
% Commande qui limite l?affichage à (xmin,ymin) et (xmax,ymax)
\newcommand {\fenetre}
{\clip (\xmin,\ymin) rectangle (\xmax,\ymax);}

%Exemple d'utilisation

%\begin{center}
%\begin{tikzpicture} [xmin=-2,xmax=2,ymin=0,ymax=5]
%\grille{1} \axes \fenetre
%\draw plot[smooth] (\x,\x^2);
%\end{tikzpicture}
%\end{center}

%style pour la perspective cavalière française
%voir Tikz pour l'impatient page 68
\tikzset{math3d/.style=
{x= {(-0.353cm,-0.353cm)}, z={(0cm,1cm)},y={(1cm,0cm)}}}

%%%%%%%Symbole pour code calculatrice%%%%%%

%Flèche remplie pour défilement de menu

\newcommand{\flechefillright}{
\begin{tikzpicture}[scale=0.15] \fill (0,0)--(2,1)--(0,2)--cycle;
\end{tikzpicture}}

%%%%%%%%%%%%%Symboles pour calculatrice Casio%%%%
\newcommand{\execasio}{\Pisymbol{psy}{191}} %Retour chariot
\newcommand{\dispcasio}{\begin{pspicture}(.1,.1)\pspolygon*(.1,0)(.1,.1)\end{pspicture}} %Triangle « Disp »
\newcommand{\dispcasiotikz}{\begin{tikzpicture}[scale=0.2]
\fill (0,0) -- (1,0) -- (1,1) -- cycle;
\end{tikzpicture}} %Triangle « Disp »
%


%%%%%%%%%%%%%%%%%%%Présentation de codes sources%%%%%%%%%%%%%%%%%
\usepackage{listings}
%On utilise l?environnement lstlisting pour insérer
%un code source.
%En plus de l?environnement lstlisting, on peut également utiliser la
%commande \lstinline qui fonctionne comme la commande \verb, en ce
%sens qu?on peut utiliser n?importe quel caractère comme délimiteur. Enfin,
%la commande \lstinputlisting permet de charger un code source depuis
%un fichier externe.
%Il y a deux manières de préciser des options : soit via l?option de l?envi-
%ronnement ou de la commande, soit en utilisant la commande \lstset
%qui permet de définir des options de manière globale.

\lstset{ %
  language=Python,                % the language of the code
  basicstyle=\ttfamily,           % the size of the fonts that are used for the code
  %numbers=left,                   % where to put the line-numbers
  numberstyle=\tiny,  % the style that is used for the line-numbers
  %stepnumber=2,                   % the step between two line-numbers. If it's 1, each line 
                                  % will be numbered
  %numbersep=5pt,                  % how far the line-numbers are from the code
  backgroundcolor=\color{white},      % choose the background color. You must add \usepackage{color}
  showspaces=false,               % show spaces adding particular underscores
  showstringspaces=false,         % underline spaces within strings
  showtabs=false,                 % show tabs within strings adding particular underscores
  frame=single,                   % adds a frame around the code
  rulecolor=\color{black},        % if not set, the frame-color may be changed on line-breaks within not-black text (e.g. comments (green here))
  tabsize=4,                      % sets default tabsize to 2 spaces
  captionpos=b,                   % sets the caption-position to bottom
  breaklines=true,                % sets automatic line breaking
  breakatwhitespace=false,        % sets if automatic breaks should only happen at whitespace
  %title=\lstname,                   % show the filename of files included with \lstinputlisting;
                                  % also try caption instead of title
  breakindent=1cm,
  keywordstyle=\color{blue},          % keyword style
  commentstyle=\color{red},       % comment style
  %stringstyle=\ttfamily\color{green},         % string literal style
  escapeinside={\%*}{*)},            % if you want to add LaTeX within your code
  morekeywords={*,...},              % if you want to add more keywords to the set
  deletekeywords={...}              % if you want to delete keywords from the given language
  upquote=true,columns=flexible,
xleftmargin=1cm,xrightmargin=1cm,
 inputencoding=utf8,			%Les lignes qui suivent sont pour le codage utf8
  extendedchars=true,
  literate=%
            {é}{{\'{e}}}1
            {è}{{\`{e}}}1
            {ê}{{\^{e}}}1
            {ë}{{\¨{e}}}1
            {û}{{\^{u}}}1
            {ù}{{\`{u}}}1
            {â}{{\^{a}}}1
            {à}{{\`{a} }}1
            {î}{{\^{i}}}1
            {ô}{{\^{o}}}1
            {ç}{{\c{c}}}1
            {Ç}{{\c{C}}}1
            {É}{{\'{E}}}1
            {Ê}{{\^{E}}}1
            {À}{{\`{A}}}1
            {Â}{{\^{A}}}1
            {Î}{{\^{I}}}1
}

\lstdefinestyle{rond}{
  numbers=none,
  frameround =tttt
}

\lstdefinestyle{compil}{
  numbers=none,
  backgroundcolor=\color{gristclair}
}
%\lstset{language=Python,basicstyle=\small , frame=single,tabsize=4,showspaces=false,showtabs=false,showstringspaces=false,numbers=left,numberstyle=\tiny , extendedchars=true}

%%%%%%%%%%%AmsMaths%%%%%%
\usepackage{amsmath,amsfonts,amssymb}
\usepackage{pifont,fourier}
\usepackage{ bclogo}


%%%%%Commande \DeclareMathOperator pour définir de nouveaux opérateurs (en lettres romaines droites)%%%%%
%\DeclareMathOperator{\sh}{sh}
%\DeclareMathOperator{\ch}{ch}

%%%%%%%%%%%%%%%%%%Maths divers%%%%%%%%%%%%%%%%%%%%%%%%%
%Delimiteurs
\newcommand{\delim}[3]{\raise #1\hbox{$\left #2\vbox to #3{}\right.$}}


%%%%%%%%%%%%%Nombres%%%%%%%%%%%%%%%%

%Ensemble prive de...
%\newcommand{\prive}{\boi}%{\backslash}

%Ensembles de nombres%%%%%%%%%%%%%%%%%
\newcommand{\R}{\mathbb{R}}
\newcommand{\N}{\mathbb{N}}
\newcommand{\D}{\mathbb{D}}
\newcommand{\Z}{\mathbb{Z}}
\newcommand{\Q}{\mathbb{Q}}
\newcommand{\C}{\mathbb{C}}
\newcommand{\df}{~\ensuremath{]0;+\infty[}~}
\newcommand{\K}{\mathbb{K}}

%%%%%%%%Arithmetique%%%%%%%%%%
%PGCD, PPCM
\newcommand{\PGCD}{\mathop{\rm PGCD}\nolimits}
\newcommand{\PPCM}{\mathop{\rm PPCM}\nolimits}

%Intervalles
\newcommand{\interoo}[2]{]#1\, ;\, #2[}
\newcommand{\Interoo}[2]{\left]#1\, ;\, #2\right[}
\newcommand{\interof}[2]{]#1\, ;\, #2]}
\newcommand{\Interof}[2]{\left]#1\, ;\, #2\right]}
\newcommand{\interfo}[2]{[#1\, ;\, #2[}
\newcommand{\Interfo}[2]{\left[#1\, ;\, #2\right[}
\newcommand{\interff}[2]{[#1\, ;\, #2]}
\newcommand{\Interff}[2]{\left[#1\, ;\, #2\right]}
%\newcommand\interentiers #1#2{[\! [#1\, ;\, #2]\! ]}
\newcommand{\interentiers}[2]{\llbracket #1\, ;\, #2\rrbracket}
%


%%%%%%%%%%%%%%Nombres complexes%%%%%

\newcommand{\ic}{\text{i}}
%\newcommand{\I}{\text{i}}
\newcommand{\im}[1]{\text{Im}\left(#1\right)}
\newcommand{\re}[1]{\text{Re}\left(#1\right)}
\newcommand{\Arg}[1]{\text{arg}\left(#1\right)}
\newcommand{\Mod}[1]{\left[#1\right]}
%Parties entière, réelle, imaginaire, nombre i
\newcommand{\ent}[1]{\text{E}\left(#1\right)}
\renewcommand{\Re}{\mathop{\rm Re}\nolimits}
\renewcommand{\Im}{\mathop{\rm Im}\nolimits}
\renewcommand{\i}{\textrm{i}}

%%%%%%%%%%%Probabilites et statistiques%%%%%
\newcommand{\loibinom}[2]{\mathcal{B}\left(#1\ ; \ #2 \right)}
\newcommand{\loinorm}[2]{\mathcal{N}\left(#1\ ; \ #2 \right)}
\newcommand{\loiexp}[1]{\mathcal{E}\left(#1\right)}
\newcommand{\proba}[1]{\mathbb{P}\big(#1\big)}
\newcommand{\probacond}[2]{\mathbb{P}_{#2}\big(#1\big)}
\newcommand{\esperance}[1]{\mathbb{E}\left(#1\right)}
\newcommand{\variance}[1]{\mathbb{V}\left(#1\right)}
\newcommand{\ecart}[1]{\sigma\left(#1\right)}
\newcommand{\dnormx}{\frac{1}{\sqrt{2\pi}} \text{e}^{-\frac{x^2}{2}}}
\newcommand{\dnormt}{\frac{1}{\sqrt{2\pi}} \text{e}^{-\frac{t^2}{2}}}
\newcommand{\nbalea}[2]{\reinitrand[first=#1, last=#2, counter=num]  \rand $\thenum$}  %retourne un entier aleatoire antre les bornes #1 et #2 comprises
%Covariance
\newcommand{\cov}{\mathop{\rm cov}\nolimits}
%


%%%%%%%%%%Analyse%%%%%%%%%%%

%%%%%%%%%%%Courbe%%%%%%%%%%%%
\newcommand{\courbe}[1]{\ensuremath{\mathcal{C}_{#1}}}

%%%%%%%Fonction exponentielle%%%%%
\newcommand{\fe}{~fonction exponentielle~}
\newcommand{\e}{\text{e}}

%Fonction cotangente
\newcommand{\cotan}{\mathop{\rm cotan}\nolimits}
%%%%%%%%%%%%%%%%%%%%%%%%%%%%%%%%%%%%%%%%%
%
%Fonctions hyperboliques
\newcommand{\ch}{\mathop{\rm ch}\nolimits}
\newcommand{\sh}{\mathop{\rm sh}\nolimits}


%%%%%%%%%%%%%%Limites%%%%%%
\newcommand{\limite}[2]{\lim\limits
_{x \to #1} #2}
\newcommand{\limitesuite}[1]{\lim\limits
_{n \to +\infty} #1}
\newcommand{\limiteg}[2]{\lim\limits
_{\substack{x \to #1 \\ x < #1 }} #2}
\newcommand{\limited}[2]{\lim\limits
_{\substack{x \to #1 \\ x > #1 }} #2}

%%%%%%%%%%Continuité%%%%%%%%%%%
\newcommand{\TVI}{théorème des valeurs intermédiaires}

%%%%%%%%%%%Suites%%%%%%%%%%%%
\newcommand{\suite}[1]{\ensuremath{\left(#1_{n}\right)}}
\newcommand{\Suite}[2]{\ensuremath{\left(#1\right)_{#2}}}
%

%%%%%%%%%%%%%%%Calcul intégral%%%%%%
\newcommand{\dx}{\ensuremath{\text{d}x}}		% dx
\newcommand{\dt}{\ensuremath{\text{d}t}}		% dt
\newcommand{\dtheta}{\ensuremath{\text{d}\theta}}		% dtheta
\newcommand{\dy}{\ensuremath{\text{d}y}}		% dy
\newcommand{\dq}{\ensuremath{\text{d}q}}		% dq

%%%Intégrale%%%
\newcommand{\integralex}[3]{\int_{#1}^{#2} #3 \ \dx}
\newcommand{\integralet}[3]{\int_{#1}^{#2} #3 \ \dt}
\newcommand{\integraletheta}[3]{\int_{#1}^{#2} #3 \ \dtheta}

%%%%%Equivalent%%
\newcommand{\equivalent}[1]{\build\sim_{#1}^{}}

%o et O%%%%
\renewcommand{\o}[2]{\build o_{#1\to #2}^{}}
\renewcommand{\O}[2]{\build O_{#1\to #2}^{}}



%%%%%%%%%%%%%%%Geometrie%%%%%%%%%%%%%%%%%%%%%%%

%%%%%%%%%%%%%%%Reperes%%%%%%%%%%%%%%
\def\Oij{\ensuremath{\left(\text{O},~\vect{\imath},~\vect{\jmath}\right)}}
\def\Oijk{\ensuremath{\left(\text{O},~\vect{\imath},~ \vect{\jmath},~ \vect{k}\right)}}
\def\Ouv{\ensuremath{\left(\text{O},~\vect{u},~\vect{v}\right)}}
\renewcommand{\ij}{(\vec\imath\, ;\vec\jmath\,)}
\newcommand{\ijk}{(\vec\imath\, ;\vec\jmath\, ;\vec k\,)}
\newcommand{\OIJ}{(O\,;\, I\,;\, J\,)}
\newcommand{\repere}[3]{\big(#1\, ;\,\vect{#2} ;\vect{#3}\big)}
\newcommand{\reperesp}[4]{\big(#1\, ;\,\vect{#2} ;\vect{#3} ;\vect{#4}\big)}

%%%%%%%%%Coordonnees%%%%%%%%%%%%%%
\newcommand{\coord}[2]{(#1\, ;\, #2)}
\newcommand{\bigcoord}[2]{\big(#1\, ;\, #2\big)}
\newcommand{\Coord}[2]{\left(#1\, ;\, #2\right)}
\newcommand{\coordesp}[3]{(#1\, ;\, #2\, ;\, #3)}
\newcommand{\bigcoordesp}[3]{\big(#1\, ;\, #2\, ;\, #3\big)}
\newcommand{\Coordesp}[3]{\left(#1\, ;\, #2\, ;\, #3\right)}
\newcommand{\Vcoord}[3]{\begin{pmatrix} #1 \\ #2 \\ #3 \end{pmatrix}}
%Symboles entre droites
%\newcommand{\paral}{\sslash}
\newcommand{\paral}{\mathop{/\!\! /}}
%

%%%%%%%%%Produit scalaire, Angles%%%%%%%%%%
\newcommand{\scal}[2]{\vect{#1} \, \cdot \, \vect{#2}}
\newcommand{\Angle}[2]{\left(\vect{#1} \, , \, \vect{#2}\right)}
\newcommand{\Anglegeo}[2]{\left(\widehat{\vect{#1} \, ; \, \vect{#2}}\right)}
\renewcommand{\angle}[1]{\widehat{#1}}
\newcommand{\anglevec}[2]{\left(\vect {#1}\, ,\,\vect {#2} \right)}
\newcommand{\anglevecteur}[2]{(#1\, , \, #2)}
\newcommand{\Anglevec}[2]{(\vecteur{#1}\, ,\,\vecteur{#2})}
\newcommand{\prodscal}[2]{#1 \, \cdot \, #2}
%


%Arc
%\newcommand{\arc}[1]{\wideparen{#1}}
\newcommand{\arcoriente}[1]{\overset{\curvearrowright}{#1}}
%
%


%%%%%%%%%%%%%%%Normes%%%%%%%%%%%%%%%%
\newcommand{\norme}[1]{\left\| #1\right\|}
\newcommand{\normebis}[1]{\delim{2pt}{\|}{9pt}\! #1\delim{2pt}{\|}{9pt}}
\newcommand{\normetriple}[1]{\left |\kern -.07em\left\| #1\right |\kern -.07em\right\|}
\newcommand{\valabs}[1]{\big| \, #1 \, \big|}
%

%%%%%%%%%%%%%%%%%%%%%%%%%%%Degré%%%%%%
%\newcommand{\Degre}{\ensuremath{^\circ}}
%La commande \degre est déjà définie dans le package babel

%%%%%%%%%%Vecteurs%%%%%%%%%%%
\newcommand{\vect}[1]{\mathchoice%
{\overrightarrow{\displaystyle\mathstrut#1\,\,}}%
{\overrightarrow{\textstyle\mathstrut#1\,\,}}%
{\overrightarrow{\scriptstyle\mathstrut#1\,\,}}%
{\overrightarrow{\scriptscriptstyle\mathstrut#1\,\,}}}



%%%%%%%%%%%%%Algebre%%%%%%%%%%%%%%%


%%%%%%%%%%Systemes%%%%%%%%%%%
%Systemes
\newcommand{\sys}[2]{
\left\lbrace
 \begin{array}{l}
  \negthickspace\negthickspace #1\\
  \negthickspace\negthickspace #2\\
 \end{array}
\right.\negthickspace\negthickspace}
\newcommand{\Sys}[3]{
\left\lbrace
 \begin{array}{l}
  #1\\
  #2\\
  #3\\
 \end{array}
\right.}
\newcommand{\Sysq}[4]{
\left\lbrace
 \begin{array}{l}
  #1\\
  #2\\
  #3\\
  #4\\
 \end{array}
\right.}
%
%

%%%%%%%%%%%%%%%%Matrices%%%%%%%%%%%%%%%%%%
%Comatrice
\newcommand{\com}{\mathop{\rm com}\nolimits}
%
%
%Trace
\newcommand{\tr}{\mathop{\rm tr}\nolimits}
%
%
%Transposee
\newcommand{\transposee}[1]{{\vphantom{#1}}^t\negmedspace #1}
%
%
%Noyau
\newcommand{\Ker}{\mathop{\rm Ker}\nolimits}
%
%

%
%Matrices
\newcommand{\Mn}{\mathcal M_n}
\newcommand{\matrice}[4]{
\left(
 \begin{array}{cc}
  #1 & #2 \\
  #3 & #4
 \end{array}
\right)}

\newcommand{\Matrice}[9]{
\left(
 \begin{array}{ccc}
  #1 & #2 & #3\\
  #4 & #5 & #6\\
  #7 & #8 & #9
 \end{array}
\right)}
\newcommand{\Vect}[3]{
\left(\negmedspace
 \begin{array}{c}
  #1\\
  #2\\
  #3
 \end{array}\negmedspace
\right)}
\newcommand{\Ideux}{\matrice{1}{0}{0}{1}}
\newcommand{\Itrois}{\Matrice{1}{0}{0}{0}{1}{0}{0}{0}{1}}
%
%
%Determinants
\newcommand{\determinant}[4]{
\left|
 \begin{array}{cc}
  #1 & #2 \\
  #3 & #4
 \end{array}
\right|}
\newcommand{\Determinant}[9]{
\left|
 \begin{array}{ccc}
  #1 & #2 & #3\\
  #4 & #5 & #6\\
  #7 & #8 & #9
 \end{array}
\right|}

%%%%%%%%%%%%%%Calculs en Latex%%%%%%%%%%%%%

\usepackage[first=1,last=100]{lcg}  %%%%%%%générer des nombres pseudo aléatoires
%%%%
\usepackage{calc} %   pour faire des calculs%%


%%%%Mise en page%%%%%
\usepackage{fancybox}
\usepackage{lastpage}
\usepackage{hyperref}
%À mettre dans le préambule pour faire apparaitre le plan à chaque section 

\AtBeginSection[ ]
{
\begin{frame}<beamer>
\frametitle{Plan}
\tableofcontents[currentsection]
\end{frame}
}

%%%%%%%¨Puces%%%%%%%%%%%%
\usepackage{enumerate}


%%%%%%%%%%%%Graphiques%%%%%%%%%%%%%%
\usepackage{graphicx,pgf}				
\usepackage{pstricks,pst-plot,pst-text,pst-tree,pst-eps,pst-node,pst-math,pstricks-add}
 


%%%%%%Numérotation des automatismes%%%%%%
\newcounter{autocompteur}
\setcounter{autocompteur}{0}
\newcommand{\automatisme}[1]{\addtocounter{autocompteur}{1}\frametitle{Automatisme  \theautocompteur  \textit{ thème : #1}}}
%%%%%%%%%%%%%%%%%%%%%%%%%%%%%%%%%%%%%%%%%%%%%%%%

%%%%%%%%%%%%%%%Francisation%%%%%%%%%%%%%%
\usepackage[french]{babel}
\frenchbsetup{StandardLists=true}

%%%%%%%%%%%%%%%%%%%%%%%%%%%%%%%%%%%%%%%%%



\begin{document}

\frame{\titlepage}




\begin{frame}
\frametitle{Table des matières}
\begin{itemize}
	\item \hyperlink{capacite2}{Capacité 2}
    \item \hyperlink{capacite3}{Capacité 3}
    \item \hyperlink{algo1}{Algorithmique 1}
    \item \hyperlink{capacite4}{Capacité 4}
    \item \hyperlink{capacite5}{Capacité 5}
    \item \hyperlink{capacite6}{Capacité 6}
    \item \hyperlink{capacite7}{Capacité 7}
\end{itemize}

\end{frame}

\begin{frame}
\frametitle{Capacité 2 : Questions 1 et 2}
\label{capacite2}
\begin{enumerate}
    \item \textbf{Question 1} : \textit{Démontrer que pour tout réel $x$ et  tout entier naturel $n$, on a $\exp(x)^{-n}=\exp(-nx)$}.
    
 Pour tout réel $x$  et  tout entier naturel $n$, on a $\exp(x)^{-n}=\frac{1}{\left(\exp(x)\right)^{n}}= \frac{1}{\exp(nx)}=\exp(-nx)$
 
 
	\item \textbf{Question 2} : {\itshape Soit $a$ un réel, calculer les expressions $A=\exp(a) \times \exp(2-a)$, $B=\left(\exp(a)+\exp(-a)\right)^{2}$ }
	
	\begin{itemize}
\item $A=\exp(a) \times \exp(2-a)=\exp(a+2-a)=\exp(2)$
\item $B=\left(\exp(a)+\exp(-a)\right)^{2}=\exp(2a)+2\exp(a)\exp(-a)+\exp(-2a)=\exp(a)+2\exp(0)+\exp(-a)=\exp(a)+\exp(-a)+2$
		
	\end{itemize}
	
	

\end{enumerate}
\end{frame}


\begin{frame}
\frametitle{Capacité 2 : Question 3 Partie 1}

 \textbf{Question 3} : {\itshape Démontrer chacune des égalités suivantes :
\begin{enumerate}
	\item[1] Pour tout réel $x$, \: $\frac{\exp(2x)-1}{\exp(2x)+1}=\frac{\exp(x)-\exp(-x)}{\exp(x)+\exp(-x)}$.
\end{enumerate} 
}


Pour tout réel $x$, \: 
\begin{align*}
\frac{\exp(2x)-1}{\exp(2x)+1}&=\frac{\exp(-x)\left( \exp(2x)-1\right)}{\exp(-x)\left( \exp(2x)+1\right)} \\
\intertext{On utilise la relation $\exp(x)\exp(-x)=1$ en multipliant par $\exp(-x)$ numérateur et dénominateur, cela permet de changer les $1$ en $\exp(-x)$ et les $\exp(2x)$ en $\exp(x)$}
 \frac{\exp(2x)-1}{\exp(2x)+1}&=\frac{\exp(-x+2x)-\exp(-x)}{\exp(-x+2x)+\exp(-x)}\\
  \frac{\exp(2x)-1}{\exp(2x)+1}&=\frac{\exp(x)-\exp(-x)}{\exp(x)+\exp(-x)}
\end{align*}
L'égalité est démontrée.
\end{frame}



\begin{frame}
\frametitle{Capacité 2 : Question 3 Partie 2}

 \textbf{Question 3} : {\itshape Démontrer chacune des égalités suivantes :
\begin{enumerate}
	\item[2] Pour tout réel $x$, \:  $4-\frac{4}{1+\exp(x)}= \frac{4}{1+\exp(-x)}$.

\end{enumerate} 
}

On utilise la même technique que dans la question précédente. Pour tout réel $x$, \: 
\begin{align*}
4-\frac{4}{1+\exp(x)}&=4 - \frac{4\exp(-x)}{(1+\exp(x))\exp(-x)} \\
4-\frac{4}{1+\exp(x)}&=4 - \frac{4\exp(-x)}{\exp(-x)+ 1} \\
\intertext{on met sur le même dénominateur}
4-\frac{4}{1+\exp(x)}&= \frac{4(\exp(-x)+ 1  ) - 4\exp(-x)}{\exp(-x)+ 1} = \frac{4}{\exp(-x)+ 1}\\
\end{align*}
L'égalité est démontrée.
\end{frame}

\begin{frame}
\frametitle{Capacité 2 : Question 3 Partie 2}

 \textbf{Question 3} : {\itshape Démontrer chacune des égalités suivantes :
\begin{enumerate}
	\item[3] Pour tout réel $x$, \:  $\left(\exp(x)+ \exp(-x)\right)^2-\left(\exp(x) - \exp(-x)\right)^2=4$.
\end{enumerate} 
}

Il suffit de développer. Pour tout réel $x$, \: 
\begin{multline*}
\left(\exp(x)+ \exp(-x)\right)^2-\left(\exp(x) - \exp(-x)\right)^2 =  \\
               \exp(2x) + 2 \exp(x)\exp(-x)+\exp(-2x) \\
               -\left(\exp(2x)-2\exp(x)\exp(-x)+\exp(-2x)\right)
\end{multline*}

\begin{multline*}
\left(\exp(x)+ \exp(-x)\right)^2-\left(\exp(x) - \exp(-x)\right)^2 = \\
\exp(2x) + 2 \exp(0)+\exp(-2x)-\left(\exp(2x)-2\exp(0)+\exp(-2x)\right)
\end{multline*}
Les termes en $\exp(2x)$ et $\exp(-2x)$ se simplifient :
\begin{equation*}
\left(\exp(x)+ \exp(-x)\right)^2-\left(\exp(x) - \exp(-x)\right)^2 = 4
\end{equation*}
L'égalité est démontrée.
\end{frame}

 
 
 
\begin{frame}[fragile]
\label{algo1}
\frametitle{Algorithmique 1 Partie 1}

la suite $\left(u_{n}\right)$ définie pour tout entier $n \geqslant 1$ par :
\begin{equation*}
u_{n}=1+\frac{1}{1} + \frac{1}{1 \times 2} + \frac{1}{1 \times 2 \times 3} + \ldots + \frac{1}{1 \times 2 \times 3 \times 4 \times \ldots \times n}
\end{equation*}
Euler a démontré qu'elle converge vers $\text{e}$.

Complétons la fonction algorithmique \texttt{u(n)} ci-dessous et son implémentation en $\texttt{Python}$ pour qu'elle retourne $u_{n}$.
	
	
\begin{center}
{\ttfamily 
\begin{tabular}{|l|}\hline 
Fonction u(n):\\
\hspace{0.5cm}s $\longleftarrow$ 1\\
\hspace{0.5cm}d $\longleftarrow$ 1\\
\hspace{0.5cm}Pour k allant de 1 à n \\
\hspace{1 cm}d $\longleftarrow$ d $\times$ k \\
\hspace{1 cm}s  $\longleftarrow$ s + 1 / d\\
\hspace{0.5cm}Retourne s \\
\hline
\end{tabular}
}
\end{center}

\end{frame}


 
\begin{frame}[fragile]
\frametitle{Algorithmique 1 Partie 2}
Voici  l'implémentation en $\texttt{Python}$ pour qu'elle retourne $u_{n}$.
	

\begin{center}
\begin{lstlisting}[style=rond]
def u(n):
	s = 1
	d = 1
	for k in range(1, n + 1):
		d = d * k
		s = s + 1 / d
	return s		
\end{lstlisting}
\end{center}

\end{frame}

\begin{frame}[fragile]
\frametitle{Algorithmique 1 Partie 3}
$2,718281$ est une valeur approchée avec 6 décimales exactes du nombre d'Euler  $\text{e}$. Modifions la fonction \texttt{Python} pour qu'elle retourne le plus petit entier $n$ tel que $\valabs{u_{n}-2,718281} < 10^{-6}$.
Il s'agit d'un algorithme de seuil.
	

\begin{center}
\begin{lstlisting}[style=rond]
def seuilU():
	s = 1
	d = 1
	n = 0
	while abs(s - 2.718281) >= 10 ** (-6):
		n = n + 1 
		d = d * n
		s = s + 1 / d
	return n		
\end{lstlisting}
\end{center}

\end{frame}


\begin{frame}
\frametitle{Capacité 3}
\label{capacite3}

La fonction exponentielle est strictement croissante sur $\R$. 


\begin{enumerate}
	\item $6<7$ donc par croissance de la fonction exponentielle, on a $\e^{7} > \e^{6} \Longleftrightarrow \e^{7} > \left(\e^{2}\right)^{3}$
	\item  $-4>-6$, donc par croissance de la fonction exponentielle, on a $\e^{-4} > \e^{-6} \Longleftrightarrow  \e^{-4} > \left(\e^{-3}\right)^{2}$
	\item $\forall x<0$, on a $-x>0>x$, donc par croissance de la fonction exponentielle $\e^{-x} > 1 > \e^{x}$
\end{enumerate}




\end{frame}

\begin{frame}
\frametitle{Capacité 4 Partie 1}

Soit la fonction $f$ définie sur l'intervalle $\Interff{-1}{2}$ par $f(x)=(-x+2)\text{e}^{x}$.

\begin{enumerate}
	\item $f$ est dérivable sur $\Interff{-1}{2}$ comme produit de deux fonctions dérivables définies par  $u(x)=-x+2$  et $v(x)=\text{e}^{x}$.
	
	On a $u'(x)=-1$ et $v'(x)=\text{e}^{x}$ et d'après une formule du cours, on a $f'=u'v+uv'$.
	
	On en déduit que $f'(x)=-\text{e}^{x} + (-x+2)\text{e}^{x}=\text{e}^{x}(-1-x+2)=\text{e}^{x}(1-x)$
	
\item D'abord, on résout une équation :
$$f'(x)=0 \Longleftrightarrow \text{e}^{x}(1-x) = 0 \Longleftrightarrow 1 - x = 0 \Longleftrightarrow 1 = x$$

Ensuite, on résout une inéquation :
$$f'(x)>0 \Longleftrightarrow \text{e}^{x}(1-x)> 0 \Longleftrightarrow 1 - x > 0 \Longleftrightarrow 1 > x$$

On a utilisé par deux fois la propriété : $\forall x, \, \text{e}^{x}>0$.

\end{enumerate}
\end{frame}


\begin{frame}
\frametitle{Capacité 4 Partie 2}

Soit la fonction $f$ définie sur l'intervalle $\Interff{-1}{2}$ par $f(x)=(-x+2)\text{e}^{x}$.

\begin{enumerate}

	
\item[2] D'abord, on résout une équation :
$$f'(x)=0 \Longleftrightarrow \text{e}^{x}(1-x) = 0 \Longleftrightarrow 1 - x = 0 \Longleftrightarrow 1 = x$$

Ensuite, on résout une inéquation :
$$f'(x)>0 \Longleftrightarrow \text{e}^{x}(1-x)> 0 \Longleftrightarrow 1 - x > 0 \Longleftrightarrow 1 > x$$

On a utilisé par deux fois la propriété : $\forall x, \, \text{e}^{x}>0$.

On en déduit que $f'(x)<0$ sur l'intervalle $\Interff{-1}{2}$, $f'(x)=0$ en $x=1$ et $f'(x)>0$ sur l'intervalle $\Interff{-1}{2}$.

Ainsi la fonction $f$ est strictement croissante sur $\Interff{-1}{1}$, atteint un maximum en $1$ et strictement décroissante sur $\Interff{1}{2}$.
\end{enumerate}
\end{frame}


\begin{frame}
\frametitle{Capacité 4 Partie 3}

\begin{center}
\includegraphics[scale=0.3]{capacite4.png}
\end{center}

\begin{enumerate}

	
\item[3] D'après la graphique, la largeur de la plaque est le maximum de la fonction $f$ sur l'intervalle $\Interff{-1}{2}$, soit  $f(1)=\text{e}^{1}=\text{e}$.

Ainsi l'aire de la plaque est égale à $L \times l = (2 - (-1))\times \text{e}=3\text{e}$.
\end{enumerate}
\end{frame}

\begin{frame}
\frametitle{Capacité 5 Equations}
\label{capacite5}

Résolution d'équations avec l'exponentielle : on utilise la propriété $\text{e}^{a}=\text{e}^{b} \Leftrightarrow a = b$.

\begin{enumerate}

	\item $\text{e}^{3x-1}=1 \Leftrightarrow \text{e}^{3x-1}=\text{e}^{0} \Leftrightarrow x = \frac{1}{3} $
		\item $\text{e}^{x^{2}+x}=\text{e} \Leftrightarrow x^{2}+x = 1 \Leftrightarrow  x^2+x-1=0$
		On résout cette équation du second degré (discriminant$\Delta= 5$) et on trouve que :
		$\text{e}^{x^{2}+x}=\text{e} \Leftrightarrow \boxed{x = \frac{-1-\sqrt{5}}{2} \text{ ou }  \frac{-1+\sqrt{5}}{2}}   $
		\item  Notons (E) l'équation  $\frac{\left(\text{e}^{x}\right)^{2}\times \text{e}^{x^2}}{\left(\text{e}^{x}\right)^{4}}=\text{e}^{3}$
$(E) \Leftrightarrow \text{e}^{2x+x^{2}-4x}=\text{e}^{3} \Leftrightarrow x^{2}-2x-3=0$
On résout cette équation du second degré (discriminant $\Delta= 16$) et on trouve que :
$(E) \Leftrightarrow \boxed{x = -1 \text{ ou } x = 3  }$
\end{enumerate}
\end{frame}



\begin{frame}
\frametitle{Capacité 5 Inéquations}
\label{capacite5}

Résolution d'inéquations avec l'exponentielle : on utilise la propriété $\text{e}^{a} \leqslant \text{e}^{b} \Leftrightarrow a \leqslant b$.

\begin{enumerate}

\item $\text{e}^{-x} \geqslant 1 \Leftrightarrow \text{e}^{-x} \geqslant \text{e}^{0}  \Leftrightarrow -x \geqslant 0   \Leftrightarrow x \leqslant 0 $.
L'ensemble des solutions est $\boxed{\mathcal{S} = \Interof{-\infty}{0}}$.
		\item $\text{e}^{-2x} > \text{e}^{x+3} \Leftrightarrow -2x > x + 3 \Leftrightarrow -1>x$
L'ensemble des solutions est $\boxed{\mathcal{S} = \Interoo{-\infty}{-1}}$.
		\item  $(I) \Leftrightarrow\text{e}^{x^2} -\left(\text{e}^{x}\right)^{2} \leqslant 0 \Leftrightarrow \text{e}^{x^2} \leqslant \text{e}^{2x}$
		 $(I) \Leftrightarrow \text{e}^{x^2} \leqslant \text{e}^{2x} \Leftrightarrow x^2 \leqslant 2x \Leftrightarrow x(x-2) \leqslant 0$
D'après la règle du signe d'un trinôme, l'ensemble des solutions est 	$\boxed{\mathcal{S} = \Interff{0}{2}}$. 
		\item $(I) \Leftrightarrow \text{e}^{2x} -\text{e}^{x} > 0 > 0 \Leftrightarrow  \text{e}^{x} \left(\text{e}^{x} - 1 \right)>0$
Pour tout réel $x$, on a $\text{e}^{x}>0$, donc $(I)\Leftrightarrow  \text{e}^{x} - 1 	\Leftrightarrow \text{e}^{x} > \text{e}^{0} \Leftrightarrow  x > 0$.  L'ensemble des solutions est $\boxed{\mathcal{S} = \Interoo{0}{+\infty}}$.
		\end{enumerate}
\end{frame}


\begin{frame}
\frametitle{Capacité 6 Question 1 a)}
\label{capacite6}

Soit la fonction $f$ définie sur $\R$ par $f(t)=\text{e}^{-1,5t}$ et $\courbe{f}$ sa courbe représentative.

\begin{itemize}
\item $f$ est dérivable sur $\R$  et pour tout réel $t$, on a $f'(t)=-1,5\text{e}^{-1,5t}$.
\item Pour tout réel $t$, on a $\text{e}^{-1,5t}>0$ et $-1,5<0$ donc $f'(t)<0$ donc $f$ est strictement décroissante sur $\R$.
\end{itemize}

\begin{center}
\includegraphics[scale=0.3]{capacite6.png}
\end{center}


\end{frame}



\begin{frame}
\frametitle{Capacité 6 Question 1 b)}


Soit la fonction $g$ définie sur $\R$ par $g(t)=\text{e}^{0,5t}$ et $\courbe{f}$ sa courbe représentative.

\begin{itemize}
\item $g$ est dérivable sur $\R$  et pour tout réel $t$, on a $g'(t)=0,5\text{e}^{0,5t}$.
\item Pour tout réel $t$, on a $\text{e}^{0,5t}>0$ et $0,5>0$ donc $g'(t)>0$ donc $g$ est strictement croissante sur $\R$.
\end{itemize}

\begin{center}
\includegraphics[scale=0.3]{capacite6bis.png}
\end{center}


\end{frame}


\begin{frame}
\frametitle{Capacité 6 Question 2)}


	\begin{itemize}
		\item $f:x \mapsto \text{e}^{2x}$ a pour courbe  $\mathcal{C}_{3}$
		\item $g:x \mapsto \text{e}^{-\frac{x}{3}}$ a pour courbe $\mathcal{C}_{2}$
		\item $h:x \mapsto \text{e}^{\frac{x}{3}}$ a pour courbe $\mathcal{C}_{4}$
		\item $k:x \mapsto \text{e}^{-2x}$ a pour courbe $\mathcal{C}_{1}$
	\end{itemize}

\begin{center}
\includegraphics[scale=0.3]{capacite6bisbis.png}
\end{center}


\end{frame}

\begin{frame}
\label{capacite7}
\frametitle{Capacité 7 Question 1)}


Un capital de $1000$ euros est placé le $1^{\text{er}}$ Janvier $2019$ au taux fixe de $1,4 \%$.

On note $c_{n}$ le capital au premier Janvier $2019+n$. 

\begin{itemize}
	\item Pour tout entier naturel $n$, on a $c_{n+1}=\left(1 + \frac{1,4}{100}\right)c_{n}=1,014 c_{n}$.
	
	On en déduit que la suite $\suite{c}$ est géométrique de raison $1,014$.
	
	\item Pour tout entier naturel $n$ on a : $c_{n+1}-c_{n}= = 1,014 c_{n} - c_{n}=0,014c_{n}$.
	
	On en déduit que l'augmentation du capital est proportionnelle au capital.
	
	\item Il s'agit d'une croissance exponentielle.
	
	
	
\end{itemize}
\end{frame}



\begin{frame}
\label{capacite7}
\frametitle{Capacité 7 Question 2)}


On veut modéliser l'évolution du capital par une fonction $f$ dérivable  sur $\Interfo{0}{+\infty}$ telle que le taux d'évolution instantané du capital $f'(t)$ est proportionnel au capital $f(t)$ selon une relation analogue à celle vérifiée par la suite $\suite{c}$. On recherche donc une fonction $f$ vérifiant l'équation $f'=0,014f$.   

\begin{itemize}
	\item Soit la $f$ définie sur $\Interfo{0}{+\infty}$ par $f(t)=k \text{e}^{0,014t}$.

Pour tout réel $t \geqslant 0$, $f'(t)=	0,014k \text{e}^{0,014t}=	0,014f(t)$.
	\item $f(0)=1000 \Leftrightarrow k \text{e}^{0,014\times 0 }=1000  \Leftrightarrow k=1000$
	
	\item Le capital au $1^{\text{er}}$ Janvier $2030$ en utilisant la suite $\suite{c}$  est $c_{11}=1,014^{11}\times 1000 \approx 1165,24$ €.
	
		\item Le capital au $1^{\text{er}}$ Janvier $2030$ en utilisant  la fonction $f$ est $f(11)=1000 \text{e}^{0,014\times 11} \approx 1166,49 $ €. 
	
\end{itemize}
\end{frame}

		
	


\end{document}